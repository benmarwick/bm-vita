%%% A template to produce a nice-looking Curriculum Vitae.
%%% Originally from Kieran Healy <kjhealy@gmail.com>
%%% Kieran's source is at http://kjhealy.github.com/kjh-vita
%%%
%%% This source code has been adapted by 
%%% Ben Marwick <benmarwick@gmail.com> and is largely
%%% determined by the instructions here:
%%% https://admin.artsci.washington.edu/promotion-and-tenure-documentation#curriculum
%%%
%%% questions: where to put on campus presentation on scholarly and teaching topics?
%%% ------------------------------------------------------------------------
%%% Requirements for this docuemnt that are included
%%%  in (or otherwise easily obtained) a modern tex distribution:
%%% ------------------------------------------------------------------------
%%% xelatex (I'm using MiKTeX & Texmaker)
%%% fontspec.sty
%%% hyperrref.sty
%%% xunicode.sty
%%% color.sty
%%% url.sty
%%% fancyhdr.sty
%%% memoir.cls
%%% fontawesome.sty
%%% gitinfo.sty
%%% 
%%% 
%%% ------------------------------------------------------------------------
%%% Requirements from https://github.com/kjhealy/latex-custom-kjh
%%% ------------------------------------------------------------------------
%%% org-preamble-xelatex.sty
%%% memoir-article-styles.sty
%%%
%%% 
%%% ------------------------------------------------------------------------
%%% Note
%%%------------------------------------------------------------------------
%%% Because this is a hand-tweaked file, be on the look out for \medksip, 
%%% \bigskip and \newpage commands here and there, which are used to balance
%%% the layout or avoid widows & orphans, etc. You should of course add or 
%%% remove these as needed.
%%%------------------------------------------------------------------------

\documentclass[11pt,article,oneside]{memoir}   
\usepackage{microtype}
\usepackage{org-preamble-xelatex} 
\usepackage{fontawesome,url}
\usepackage{setspace}
\usepackage[footinfo]{gitinfo}



%%%------------------------------------------------------------------------
%%% Metadata
%%%------------------------------------------------------------------------

%% Change as needed.
\def\myauthor{Ben Marwick}
\def\mytitle{Vita}
\def\mycopyright{\myauthor}
\def\mykeywords{}
\def\mybibliostyle{plain}
\def\mybibliocommand{}
\def\mysubtitle{}
\def\myaffiliation{University of Washington}
\def\myaddress{Anthropology Department}
\def\myemail{bmarwick@uw.edu}
\def\myweb{http://faculty.washington.edu/bmarwick/ }
\def\myphone{(206) 552-9450}
\def\myfax{(206) 543-3285}
\def\mytwitter{@benmarwick}
\def\myversion{}
\def\myrevision{}


\def\myaffiliation{University of Washington}
\def\myauthor{Ben Marwick}
\date{} % not used (revision control instead)
\def\mykeywords{}


%%%------------------------------------------------------------------------
%%% Document
%%%------------------------------------------------------------------------
\begin{document}

%% Choose fonts for use with xelatex
%% Using free fonts with big glyph sets for maximum flexibility  
%% http://www.linuxlibertine.org/index.php?id=2&L=1
%% http://font.ubuntu.com/

\setromanfont[Mapping={tex-text}, 
	Numbers={OldStyle},
	Ligatures={Common}]{Linux Libertine}
\setsansfont[Mapping=tex-text,
	Ligatures={Common}, 
	Colour=AA0000]{Linux Biolinum}
\setmonofont[Mapping=tex-text,Scale=0.72]{Ubuntu} 

\newfontface\scheader[SmallCapsFont={Linux Libertine},SmallCapsFeatures={Letters=SmallCaps}]{Linux Libertine}

\newfontface\addressblock[Mapping={tex-text}, 
	Numbers={OldStyle},
	Ligatures={Common}]{Linux Libertine}


%%%------------------------------------------------------------------------
%%% Local commands
%%%------------------------------------------------------------------------

%% Marginal header
%% Note: as the document goes on you may need to introduce a (gradually increasing)
%% \vspace element to keep the marginal header pleasingly aligned with the first 
%% item in the body text. Like this: \marginhead{{\vskip 0.4em}Grants}, or 
%% \marginhead{{\vskip 0.8em}Service}. Experiment as needed.
\newcommand{\marginhead}[1]{\marginpar{\textsf{{\footnotesize\vspace{-1em}\flushright #1}}}}


%% [optional] custom ampersand (font consistent with the one chosen above)
\newcommand{\amper}{{\fontspec[Scale=.95,Colour=AA0000]{Linux Libertine}\selectfont\&\,}}

%% No bullets on labels
\renewcommand{\labelitemi}{~} 

%% Custom hanging indent for vita items
\def\ind{\hangindent=1 true cm\hangafter=1 \noindent}
%\def\ind{\hangindent=18pt\hangafter=1 \noindent}
\def\labelitemi{~}
\renewcommand{\labelitemii}{~}

%%%------------------------------------------------------------------------
%%% Page layout
%%%------------------------------------------------------------------------

% These lines will insert git revision info in the footer, using the gitinfo
% package---see docs for gitinfo package for details. Comment out this line
% if you're not using git.
\pagestyle{kjh}
\thispagestyle{kjhgit}

%%%------------------------------------------------------------------------
%%% Address and contact block
%%%------------------------------------------------------------------------
\begin{minipage}[t]{2.95in}
 \flushright {\footnotesize 
 \href{http://depts.washington.edu/anthweb/}{Department of Anthropology} \\ Box 353100 \\ University of Washington  \\ \vspace{-0.05in} Seattle \textsc{wa} 98195-3100}  
  
\end{minipage}
\hfill     
%\begin{minipage}[t]{0.0in}
% dummy (needed here)
%\end{minipage}
\hfill
\begin{minipage}[t]{1.7in}
  \flushright \footnotesize  \addressblock \myphone \, \faPhone \\ 
  {\scriptsize  \texttt{\href{http://twitter.com/benmarwick}{\mytwitter}} \, \faTwitter }  \\ 
  {\scriptsize  \texttt{\href{mailto:\myemail}{\myemail}} \, \faEnvelope} \\
  {\scriptsize  \texttt{\href{\myweb}{\myweb}} \, \faGlobe}
\end{minipage}

\medskip

%% Name 
\noindent{\LARGE\scheader \textsc{bibliography}}
\reversemarginpar

\bigskip       

% Bibliography: https://admin.artsci.washington.edu/promotion-and-tenure-documentation#curriculum

% The candidate's complete bibliography should be submitted, with entries listed in full bibliographic format, including range of page numbers or number of pages. In addition, the following items should be clearly indicated or distinguished:

% the type of each publication (article, book, review, monograph, technical report, etc.)
% the principal author(s) on multi-authored publications
% those publications which have had outside peer review before acceptance for publication
% a labeling of nonpublished material as "in press" or "accepted for publication," "submitted," or "work in progress" (note that "forthcoming" is ambiguous and should not be used)

\marginhead{\sffamily {{\vskip -0.35em} peer reviewed \newline journal articles}}


\ind Lowe, K., L. Wallis, C. Pardoe, \textbf{B. Marwick}, C. Clarkson, T. Manne, M. Smith and Richard Fullagar. under review. Ground-penetrating radar and burial practices in western Arnhem Land, Australia Archaeology in Oceania  [submitted Nov 2013, awaiting peer reviews] 

\ind \textbf{Marwick, B.}, Hughes, P. P. Hiscock and M. Sullivan. under review. Boundary effects in forager site distribution in arid northeast South Australia: A case study of mobile Geographic Information Systems for intensive archaeological survey. Archaeology in Oceania  [submitted Nov 2013, awaiting peer reviews]

\ind \textbf{Marwick, B.} and V. Keopanna under review Geometric morphometry of 3D models for quantitative comparison of bronze Buddha statues in Luang Prabang, northern Lao PDR. World Archaeology [submitted Oct 2013, awaiting peer reviews]

\ind Conrad, C., H. Van Vlack, \textbf{B. Marwick}, C. Thongcharoenchaikit, R. Shoocongdej, and B. Chaisuwan 2013 Summary of vertebrate and molluscan assemblages excavated from late-Pleistocene and Holocene deposits at Khao Toh Chong Rockshelter, Krabi, Thailand. Thailand Natural History Museum Journal 7(1): 11-22

\ind \textbf{Marwick, B.} 2013. Multiple Optima in Hoabinhian flaked stone artefact palaeoeconomics and palaeoecology at two archaeological sites in Northwest Thailand Journal of Anthropological Archaeology 34(2): 553–564.

\ind Brockwell, S., \textbf{B. Marwick}, P. Bourke, P. Faulkner and R. Willan. In press 2013 Late Holocene Climate Change and Human Behavioural Variability in the Coastal Wet-Dry Tropics of Northern Australia: Evidence from a Pilot Study of Oxygen Isotopes in Molluscan Shells. Australian Archaeology 76: 21–33.

\ind Sullivan, M., T. Field, P. Hughes, \textbf{B. Marwick}, P. Przystupa, J. Feathers 2012. OSL dates that inform late phases of dune formation and human occupation near Olympic Dam in northeastern South Australia. Quaternary Australasia 29(1): 4-11.

\ind Hughes, P. P. Hiscock, M. Sullivan and \textbf{B. Marwick}. 2011. An outline of archaeological investigations for the Olympic Dam Expansion in arid northeast South Australia. Journal of the Anthropological Society of South Australia 34:21-37

\ind \textbf{Marwick, B.} and M. Gagan 2011. Late Pleistocene monsoon variability in northwest Thailand: an oxygen isotope sequence from the bivalve Margaritanopsis laosensis excavated in Mae Hong Son province. Quaternary Science Reviews 30(21-22): 3088-3098

\ind \textbf{Marwick, B.} 2010. Self-image, the long view and archaeological engagement with film: An animated case study. World Archaeology 42(3): 394–404

\ind \textbf{Marwick, B.} 2009a. Change or Decay? An interpretation of late Holocene archaeological evidence from the Hamersley Plateau, Western Australia. Archaeology in Oceania 44: 16-22

\ind \textbf{Marwick, B.}  2009b. Biogeography of Middle Pleistocene Hominins in Mainland Southeast Asia: A Review of Current Evidence. Quaternary International 202 (2009) 51–58

\ind \textbf{Marwick, B.} 2008a. Stone artefacts and recent research in the archaeology of mainland Southeast Asian hunter-gatherers. Before Farming 2008/4

\ind \textbf{Marwick, B.} 2008b. Three Styles of Darwinian Evolution in the Analysis of Stone Artefacts: Which one to use in Mainland Southeast Asia? Australian Archaeology 67: 79-86

\ind \textbf{Marwick, B.} 2008c. Beyond typologies: The reduction thesis and its implications for lithic assemblages in Southeast Asia. Bulletin of the Indo-Pacific Prehistory Association 28: 108-116

\ind \textbf{Marwick, B.} 2008d. Identifying variation in Hoabinhian lithic technology: An index of flake cortex distribution for the measurement of cobble reduction intensity. Journal of Archaeological Science 35(5): 1189-1200. (Winner, Dorothy Cameron Prize)

\ind \textbf{Marwick, B.} 2007. Approaches to Stone Artefact Archaeology in Thailand: A Historical Review. Silpakorn University International Journal 7:49-88

\ind \textbf{Marwick, B.} 2005a. The interpersonal origins of language: Social and linguistic implications of an archaeological approach to language evolution. Linguistics and the Human Sciences 1.2: 197-224.

\ind \textbf{Marwick, B.} 2005b. Element Concentrations and Magnetic Susceptibility of Anthrosols: Indicators of Prehistoric Human Occupation in the inland Pilbara, Western Australia. Journal of Archaeological Science 32: 1357-1368.

\ind \textbf{Marwick, B.} 2003. Pleistocene exchange networks as evidence for the evolution of language. Cambridge Archaeological Journal 17(1):67-81. (Winner, Dorothy Cameron Prize)

\ind \textbf{Marwick, B.} 2002a. Milly’s Cave: Evidence for Human Occupation of the Inland Pilbara during the Last Glacial Maximum. Tempus 21-33.

\ind \textbf{Marwick, B.} 2002b. Evidence of Prehistoric Occupation of the Abrolhos Islands, Western Australia. Records of the Western Australian Museum 20(4):461-464.

\bigskip

\marginhead{\sffamily {{\vskip -0.35em} peer reviewed \newline book chapters}}

\ind \textbf{Marwick, B.} 2013. Text mining and topic modeling for discovery of current issues in Anthropology using microblog content. In  Yanchang Zhao and Yonghua Cen (eds) Data Mining Applications with R Elsevier. p. 63-93.

\ind \textbf{Marwick, B.}, R. Shoocongdej, C. Thongcharoenchaikit, B. Chaisuwan, C. Khowkhiew and S. Kwak.  Hierarchies of engagement and understanding: Community engagement during archaeological excavations at Khao Toh Chong rockshelter, Krabi, Thailand. In O’Connor, S (ed.) Transcending the Culture-Nature Divide in Cultural Heritage: Views from the Asia-Pacific Region. Terra Australis, ANU E Press. 

\ind \textbf{Marwick, B.} 2012. Transmission of craft traditions in bronze Buddha manufacturing in ancient Thailand: A pilot study of the Griswold collection using cladistic techniques. In Dominik Bonatz, Andreas Reinecke, Mai Lin Tjoa-Bonatz (eds.) Connecting Empires and States: Selected Papers from the 13th Conference of the European Association of Southeast Asian Archaeologists. National University of Singapore Press, Singapore. pp. 161-178. 

\ind Mackay, A. and \textbf{B. Marwick} 2011. Costs and benefits in technological decision making under variable conditions: examples from the late Pleistocene in southern Africa. In \textbf{B. Marwick} and A. Mackay (eds.) Keeping Your Edge: Recent Approaches to the Organisation of Stone Artefact Technology. British Archaeological Reports.  pp. 119-134 

\ind \textbf{Marwick, B.} 2008. Human Behavioural Ecology and Stone Artefacts in Northwest Thailand during the Terminal Pleistocene and Holocene. Jean-Pierre Pautreau (ed) Southeast Asian Prehistory: The 11th EurASEAA. Musée de Bougon: Paris. pp. 37-49. 

\bigskip

\marginhead{\sffamily {{\vskip -0.35em} book reviews \newline (no outside \newline peer review)}}

\ind \textbf{Marwick, B.} 2009. Review of “Cultural Transmission and Material Culture: Breaking Down Boundaries.” Miriam Stark, Brenda Bowser and Lee Horne, eds. Tucson: University of Arizona Press, 2008. American Anthropologist 111(4): 540-541

\ind \textbf{Marwick, B.} 2009. Review of "Place as Occupational Histories: An Investigation of the Deflated Surface Archaeological Record of Pine Point and Langwell Stations, Western New South Wales, Australia." by Justin Shiner. Archaeopress, Oxford, (2008). Australian Archaeology 69: 82

\ind \textbf{Marwick, B.} 2007. Review of “The origins and evolution of cultures”, by Peter Richerson and Robert Boyd. Oxford University Press, London, (2004). The Australian Journal of Anthropology 18(1): 97-98.

\ind \textbf{Marwick, B.} 2006. What can archaeology do with Boyd and Richerson’s Co-evolutionary program? Review of “Not By Genes Alone”, by Peter Richerson and Robert Boyd. Chicago University Press, Chicago, (2005). The Review of Archaeology 26(2): 30-40.

\ind \textbf{Marwick, B.} 2005a. Review of “Agency uncovered: Archaeological perspectives on social agency, power, and being human, edited by Andrew Gardner. UCL Press, London, (2004). Anthropological Forum 15(2): 203-204.

\ind \textbf{Marwick, B.} 2005b. Review of “Marx’s ghost: Conversations with archaeologists”, by Thomas C. Patterson. Berg, Oxford. (2003). Anthropological Forum 15(1): 92-94

\ind \textbf{Marwick, B.} 2003a. Review of “Batavia’s graveyard. The true story of the mad heretic who lead history’s bloodiest mutiny”, by Mike Dash. London: Weidenfeld and Nicolson. (2002). Limina vol. 9: 236-240.

\ind \textbf{Marwick, B.} 2003b. Review of “A dictionary of archaeology”, edited by Ian Shaw and Robert Jameson. Oxford: Blackwell Publishers Ltd. (2002) Australian Archaeology 54: 50-51

\bigskip

\marginhead{\sffamily {{\vskip -0.35em} editor-reviewed \newline publications \newline (no outside \newline peer review)}}

\ind \textbf{Marwick, B.} and A. Mackay 2011. Keeping your edge: Recent approaches to the organisation of stone artefact technology. In \textbf{B. Marwick} and A. Mackay (eds.) Keeping Your Edge: Recent Approaches to the Organisation of Stone Artefact Technology. British Archaeological Reports.  pp. 1-4.

\ind White, J.C., Lewis, H., Bouasisengpaseuth, B., \textbf{Marwick, B.} and K. Arrell 2009. Archaeological investigations in northern Laos: New contributions to Southeast Asian prehistory. Antiquity 83(319) Project Gallery

\ind \textbf{Marwick, B.}, White, J.C., B. Bouasisengpaseuth 2009. The Middle Mekong Archaeology Project and International Collaboration in Luang Prabang, Laos. SAA Archaeological Record 9 (3) 25-27

\ind \textbf{Marwick, B.} 2006a. A Methodological Study of Technological Attributes in Hoabinhian Lithic Assemblages. In R. Shoocondej (ed.)  Social, Cultural and Environmental Dynamics in the Highlands of Pangmapha, Mae Hong Son Province: Integrated Archaeological Research into the Region. Highland Archaeology Project in Pangmapha: Bangkok. pp. 231-244. 

\ind \textbf{Marwick, B.} 2006b. Report on the 18th Congress of the Indo-Pacific Prehistory Association. Southeast Asian Archaeology Newsletter

\ind \textbf{Marwick, B.} 2004. Future directions for research into open sites and rockshelters in the inland Pilbara. Australian Association of Consulting Archaeologists Inc. Newsletter 95:14-18.


\end{document}
