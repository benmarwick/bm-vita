%%% A template to produce a nice-looking Curriculum Vitae.
%%% Originally from Kieran Healy <kjhealy@gmail.com>
%%% Kieran's source is at http://kjhealy.github.com/kjh-vita
%%%
%%% This source code has been adapted by 
%%% Ben Marwick <benmarwick@gmail.com> and is largely
%%% determined by the instructions here:
%%% https://admin.artsci.washington.edu/promotion-and-tenure-documentation#curriculum
%%%
%%% questions: where to put on campus presentation on scholarly and teaching topics?
%%% ------------------------------------------------------------------------
%%% Requirements for this docuemnt that are included
%%%  in (or otherwise easily obtained) a modern tex distribution:
%%% ------------------------------------------------------------------------
%%% xelatex (I'm using MiKTeX & Texmaker)
%%% fontspec.sty
%%% hyperrref.sty
%%% xunicode.sty
%%% color.sty
%%% url.sty
%%% fancyhdr.sty
%%% memoir.cls
%%% fontawesome.sty
%%% gitinfo.sty
%%% 
%%% 
%%% ------------------------------------------------------------------------
%%% Requirements from https://github.com/kjhealy/latex-custom-kjh
%%% ------------------------------------------------------------------------
%%% org-preamble-xelatex.sty
%%% memoir-article-styles.sty
%%%
%%% 
%%% ------------------------------------------------------------------------
%%% Note
%%%------------------------------------------------------------------------
%%% Because this is a hand-tweaked file, be on the look out for \medksip, 
%%% \bigskip and \newpage commands here and there, which are used to balance
%%% the layout or avoid widows & orphans, etc. You should of course add or 
%%% remove these as needed.
%%%------------------------------------------------------------------------

\documentclass[11pt,article,oneside]{memoir}   
\usepackage{microtype}
\usepackage{org-preamble-xelatex} 
\usepackage{fontawesome,url}
\usepackage{setspace}
\usepackage[footinfo]{gitinfo}

%%%------------------------------------------------------------------------
%%% Metadata
%%%------------------------------------------------------------------------

%% Change as needed.
\def\myauthor{Ben Marwick}
\def\mytitle{Vita}
\def\mycopyright{\myauthor}
\def\mykeywords{}
\def\mybibliostyle{plain}
\def\mybibliocommand{}
\def\mysubtitle{}
\def\myaffiliation{University of Washington}
\def\myaddress{Anthropology Department}
\def\myemail{bmarwick@uw.edu}
\def\myweb{http://faculty.washington.edu/bmarwick/ }
\def\myphone{(206) 552-9450}
\def\myfax{(206) 543-3285}
\def\mytwitter{@benmarwick}
\def\myversion{}
\def\myrevision{}


\def\myaffiliation{University of Washington}
\def\myauthor{Ben Marwick}
\date{} % not used (revision control instead)
\def\mykeywords{}


%%%------------------------------------------------------------------------
%%% Document
%%%------------------------------------------------------------------------

\begin{document}

%% Choose fonts for use with xelatex
%% Using free fonts with big glyph sets for maximum flexibility  
%% http://www.linuxlibertine.org/index.php?id=2&L=1
%% http://font.ubuntu.com/

\setromanfont[Mapping={tex-text}, 
	Numbers={OldStyle},
	Ligatures={Common}]{Linux Libertine}
\setsansfont[Mapping=tex-text,
	Ligatures={Common}, 
	Colour=AA0000]{Linux Biolinum}
\setmonofont[Mapping=tex-text,Scale=0.72]{Ubuntu} 

\newfontface\scheader[SmallCapsFont={Linux Libertine},SmallCapsFeatures={Letters=SmallCaps}]{Linux Libertine}

\newfontface\addressblock[Mapping={tex-text}, 
	Numbers={OldStyle},
	Ligatures={Common}]{Linux Libertine}


%%%------------------------------------------------------------------------
%%% Local commands
%%%------------------------------------------------------------------------

%% Marginal header
%% Note: as the document goes on you may need to introduce a (gradually increasing)
%% \vspace element to keep the marginal header pleasingly aligned with the first 
%% item in the body text. Like this: \marginhead{{\vskip 0.4em}Grants}, or 
%% \marginhead{{\vskip 0.8em}Service}. Experiment as needed.
\newcommand{\marginhead}[1]{\marginpar{\textsf{{\footnotesize\vspace{-1em}\flushright #1}}}}


%% [optional] custom ampersand (font consistent with the one chosen above)
\newcommand{\amper}{{\fontspec[Scale=.95,Colour=AA0000]{Linux Libertine}\selectfont\&\,}}

%% No bullets on labels
\renewcommand{\labelitemi}{~} 

%% Custom hanging indent for vita items
\def\ind{\hangindent=1 true cm\hangafter=1 \noindent}
%\def\ind{\hangindent=18pt\hangafter=1 \noindent}
\def\labelitemi{~}
\renewcommand{\labelitemii}{~}

%%%------------------------------------------------------------------------
%%% Page layout
%%%------------------------------------------------------------------------

% These lines will insert git revision info in the footer, using the gitinfo
% package---see docs for gitinfo package for details. Comment out this line
% if you're not using git.
\pagestyle{kjh}
\thispagestyle{kjhgit}

%%%------------------------------------------------------------------------
%%% Address and contact block
%%%------------------------------------------------------------------------
\begin{minipage}[t]{2.95in}
 \flushright {\footnotesize 
 \href{http://depts.washington.edu/anthweb/}{Department of Anthropology} \\ Box 353100 \\ University of Washington  \\ \vspace{-0.05in} Seattle \textsc{wa} 98195-3100}  
  
\end{minipage}
\hfill     
%\begin{minipage}[t]{0.0in}
% dummy (needed here)
%\end{minipage}
\hfill
\begin{minipage}[t]{1.7in}
  \flushright \footnotesize  \addressblock \myphone \, \faPhone \\ 
  {\scriptsize  \texttt{\href{http://twitter.com/benmarwick}{\mytwitter}} \, \faTwitter }  \\ 
  {\scriptsize  \texttt{\href{mailto:\myemail}{\myemail}} \, \faEnvelope} \\
  {\scriptsize  \texttt{\href{\myweb}{\myweb}} \, \faGlobe}
\end{minipage}

\medskip

%% Name 
\noindent{\LARGE\scheader \textsc{personal statement}}
\reversemarginpar

\bigskip       

% Personal statement for tenure
% http://admin.artsci.washington.edu/promotion-and-tenure-documentation#candidates

% The statement should not be longer than three to five pages.
%
% In the discussion of research or scholarship, a short essay is more effective than an annotated list of works. The candidate's research contributions might be described in the broader context of the discipline as a whole, explaining how his or her research agenda fits into the discipline and then how particular scholarly or creative contributions fit into this agenda. The essay should also include the candidate's statement of future directions and how these connect to previous and current work, in order to give a sense of the trajectory of the work.
%
% The personal statement should contain as well a discussion of the candidate's teaching experience, with an overview of the candidate's goals, a review of successes and failures, reflections on these experiences, and thoughts of what lies ahead. Lastly, the candidate should describe any significant service contributions.

% ideas... 
% http://www.insidehighered.com/advice/2010/11/10/narrative
% http://www.slideshare.net/UO-AcademicAffairs/writing-a-tenure-statement-2011


%% set paragaph breaks to be blank lines and don't indent first line
\nonzeroparskip
\setlength{\parindent}{0pt}

\bigskip       

\marginhead{\sffamily {{\vskip 0.5em}  introduction}}

I am an archaeologist interested in human behavior, technology and ecology. My approach to these themes is motivated by models and methods from the evolutionary sciences which I seek to adapt to better understand the human past. My passion as a scholar is to apply evolutionary approaches to address questions of technological variation and ecological adaptation, cultural change and cultural transmission ranging from the deep past to recent times. Unlike many archaeologists motivated by evolutionary theory, I am fascinated by the challenges of primary data collection, and my fieldwork in mainland Southeast Asia and Australia supplies the empirical content of my research. Fundamental to these activities is my deep concern for the social relevance and disciplinary integrity for archaeology. What sets my work apart from others is this combination of evolutionary theory applied to primary data collected in Southeast Asia in collaborative arrangements that prioritize training and capacity building.  

In what follows, I will describe my research trajectory since arriving at the University of Washington in 2008 and explain how the characteristics that distinguish my research – namely, its use of evolutionary approaches and its focus on archaeological problems of Southeast Asia – contribute to my teaching and service. 

I began my research career investigating questions about Australian Aboriginal hunter-gatherer adaptation during the Late Pleistocene and Holocene periods (Marwick {\href{http://hdl.handle.net/1885/42085}{2002a}}, {\href{http://dx.doi.org/10.6084/m9.figshare.765251}{2002b}}, {\href{http://faculty.washington.edu/bmarwick/PDFs/Marwick_2005_Marillana_A.pdf}{2005c}},  \href{http://faculty.washington.edu/bmarwick/PDFs/Marwick_2009_AO_Pilbara.pdf}{2009},  \href{http://faculty.washington.edu/bmarwick/PDFs/Hughes_et_al_2011_JASSA.pdf}{Hughes et al. 2011},  \href{http://faculty.washington.edu/bmarwick/PDFs/Sullivan_et_al_2012_OSL_dates_ODX.pdf}{Sullivan et al. 2012},  \href{http://faculty.washington.edu/bmarwick/PDFs/Brockwell_et_al_2013_AA.pdf}{Brockwell et al. 2013} ). Motivated by the success of combining stone artefact and geoarchaeological analyses in overcoming problems of a sparse archaeological record, I took up the challenge of  investigating prehistoric human-environment relations in tropical Southeast Asia, where the archaeological record is notoriously sparse for hunter-gatherers. My dissertation research {\href{http://dx.doi.org/10.6084/m9.figshare.765252}{(Marwick 2008)}} into the relationship between hunter-gatherer technological variation and local ecological conditions in Thailand was driven by the question of how the ecological conditions of the seasonal tropics related to decisions people made to change their stone artefact technology or change their settlement locations in the tropics. My dissertation research was also the product of a near-total immersion in an all-Thai archaeological project, an experience that has strongly shaped my approach to international collaborative research. 

%% \bigskip  
\newpage   

\marginhead{\sffamily  {{\vskip 0.3em} research \newline trajectory}}

Directly emerging from my dissertation research have been three related projects.

\begin{enumerate}

\item  I constructed a local palaeoclimate proxy from oxygen isotope values obtained from shell fish recovered from the same archaeological deposit as the stone artefacts {\href{http://faculty.washington.edu/bmarwick/PDFs/Marwick_and_Gagan_2011_QSR.pdf}{(Marwick and Gagan 2011)}}. This oxygen isotope curve provides the first evidence of a major change in conditions from the Pleistocene to the Holocene,  consistent with evidence from similar records in China. We noted the absence of a signal of the Younger Dryas (YD) in our data, a key climatic event linked to the emergence of agriculture in many parts of the world. We used this evidence to argue that agriculture in Southeast Asia most likely originated in China where there is a strong YD signal.

\item As a core member of the Middle Mekong Archaeology Project I excavated two rockshelters in the uplands of northern Laos to further investigate the question of how hunter-gatherers became farmers in mainland Southeast Asia. We predicted that if agriculture moved into Southeast Asia from China, we might find strong early signals in Laos, due to the ease of movement facilitated by the Mekong. In addition to field research we conducted extensive training of local archaeologists and museum workers  {\href{http://faculty.washington.edu/bmarwick/PDFs/Marwick_et_al_2009_MMAP.pdf}{(Marwick et al. 2009)}}. The archaeological results, noted in our \textit{Antiquity} article {\href{http://antiquity.ac.uk/projgall/white/}{(White et al 2009})}, proved to be of limited relevance to my core interest in hunter-gatherers because the excavations recovered extensively bioturbated and only very recent (Iron Age) deposits. 

\item I made a novel modification to standard behavioural ecological models  {\href{http://faculty.washington.edu/bmarwick/PDFs/Marwick_2013_JAA.pdf}{(Marwick 2013)}}. I began this analysis with the application of three standard behavioural ecological models but found that when testing the predictions of the models with archaeological data, the three models did not give a consistent indication of the main variables influencing stone artefact assemblage production. Taking inspiration from Sewall Wright's  idea of multiple optima in evolutionary biology, I revised the standard models to incorporate multiple behavioral optima and found a consistent fit with my data. 

\end{enumerate}

With my early work focused on inland and upland settings, I became curious about human adaptation in tropical coastal and island environments. Specifically, I was driven to evaluate the usefulness of my model for people with marine adaptations and to use the model as a tool to understand transitions to agriculture in regions where past sea level changes – which others have argued was a driving factor in the domestication process –  had a profound effect on the landscape. Two current projects directly relate to this question.

\begin{enumerate}

\item While a Luce/ACLS post-doctoral fellow during 2010-11, I initiated a project with Thai collaborators to explore hunter-gatherer adaptations in coastal environments through survey, excavation in southern Thailand. Together with my Thai colleagues and US graduate students, I have been active presenting the results of our analysis at professional meetings and have one scholarly paper published {\href{http://faculty.washington.edu/bmarwick/PDFs/Conrad_et_al_2013_TNHMJ.pdf}{(Conrad et al. 2013)}}. So far our data show a clear signal of subsistence behaviours focused on fresh-water resources during the late Pleistocene and early Holocene shifting to mangrove swamp resources during the later Holocene as sea levels rose.  This is a much earlier shift than previously documented, challenging earlier work that recorded this dietary shift – often linked to the transition to agriculture – much later in the Holocene. 

\item My recent work on Sulawesi, Indonesia was motivated by the question of whether my multiple optima model can be generalized to include tropical hunter-gatherer adaptations in island settings. As as co-PI with six collaborators on a project funded by the Australian Research Council (ARC), I conducted six excavations on Sulawesi during 2012-13. Two of these yielded archaeological sequences spanning the Late Pleistocene to recent times, providing data relevant to evaluating my model. We have presented some preliminary data at conferences and the detailed analyses are currently in progress. 

\end{enumerate}

To better understand these questions of human adaptation and the specific historical trajectories leading up to major events such as the transition to agriculture, I have found it necessary to broaden my inquiry to investigate the longer term evolutionary processes of humans in Southeast Asia and Australia.  My \textit{ Quaternary International} article surveys the available evidence for hominin colonization of Southeast Asia and details three models that fit the data {\href{http://faculty.washington.edu/bmarwick/PDFs/Marwick_2009_QI.pdf}{(Marwick 2009b)}}. I have been engaged in testing these models with three current projects. 

\begin{enumerate}

\item I am a co-PI with five colleagues on an ARC-funded project to investigate modern human origins and early behavioural complexity in Australia and Southeast Asia. The aim of this project is to collect and compare a large sample of early materials from three locations across Southeast Asia to Australia to more reliably date the appearance of modern humans and document the emergence of cultural diversity. In 2012 we excavated at Malakanunja II in northern Australia, one of Australia's oldest sites, in 2014 we will excavate at Jerimalai in East Timor, and in 2015 we will return to Khao Toh Chong in Thailand. I have been leading the geoarchaeological analysis and contributing to the lithic analysis of the Malakanunja II excavation. 

\item I am a co-PI with three colleagues on a Leakey Foundation grant to excavate archaeological deposits in Sumatra to investigate human colonization and adaptation relating to the Toba eruption 74 thousand years ago (fieldwork in Sumatra is scheduled for 2014). 

\item I am a co-PI with 13 colleagues on an ARC-funded project investigating hominin colonisation from India to Australia, with my specific responsibility being Myanmar and Thailand (fieldwork in Myanmar is scheduled for 2015).

\end{enumerate}

Whilst engaged in these deep time questions, I developed an interest in the potential of evolutionary archaeology to provide insights into more recent historical and political processes.  Specifically, I have been pursuing attribute-based phylogenies and morphometric analyses of bronze Buddha statues from Laos and Thailand to better understand the flow of influence between different production centers. I have published some results of the phylogenetic analysis \href{http://faculty.washington.edu/bmarwick/PDFs/Marwick_2012_Buddha_cladistics.pdf}{(Marwick 2012)} and am currently working on the morphometric analysis. 

Concurrent with my field and laboratory work I have been examining questions of archaeological engagement. As a Luce/ACLS fellow I developed a model of collaborative archaeology that has influenced my subsequent work {\href{http://faculty.washington.edu/bmarwick/PDFs/Marwick_et_al_2013_community.pdf}{(Marwick et al. 2013)}}. Specifically, I argued that community engagement can occur in three forms: syntactic (highly technical), semantic (generation of meanings) and pragmatic (resulting in practical intervention), and I list when and how each form might be best deployed. I have also investigated how popular culture engages with archaeology 


Second is my ARCHY 570 ‘Archaeology and Explanation’ class, which conducts a parallel investigation of approaches to explanation by philosophers of science and by archaeologists. The idea for this class grew out of a concern that archaeologists could be better informed about some of the basic intellectual activities that they do when making sense of the human past. My students from this class organized a session on archaeology in explanation at the 2010 Society of American Archaeology which included three senior non-UW archaeologists noted for their past philosophical contributions. I contributed a paper advocating model-based explanations, which has already been requested several times in draft form and I am currently preparing for publication.



\bigskip     

\marginhead{\sffamily teaching}

Describe your philosophy regarding the teaching and training of the next generation of scientists and, if appropriate non-scientists (for example, general education students or future K-12 teachers). Prepare a table that summarizes your teaching activity semester by semester (including course number, course title, number of students, and course evaluation information); acknowledge if the course is co-taught.

My teaching philosophy and methods are informed by evidence-based findings about US college pedagogy and benefit from the same kind of study of peer-reviewed literature as my archaeological work. My general approach is to stimulate critical thinking about the human past and is motivated by seven principles distilled by Chickering and Gamson in their 1987 a meta-analysis of 50 years of research on good teaching principles. These include: encouraging contact between students and faculty, developing reciprocity and cooperation among students, encouraging active learning, giving prompt feedback, emphasizing time on task, communicating high expectations, and respecting diverse talents and ways of learning. These principles are the foundation of my syllabus design, course scheduling, assignments and the in-person engagement I have with my students. At the level of planning individual lessons, I organize my class time according to the three factors identified by the National Research Council report “How People Learn” (2000) as critical to learning: addressing student’s preconceptions of the discipline, helping students develop a rich conceptual framework into which to place factual knowledge, and enhancing a student’s ability to monitor their learning. I also routinely consult peer-reviewed literature on data visualization, cognitive load theory and student information literacy to continuously improve my engagement with students in and out of the classroom. I am an intensive user of technology in my teaching, employing substantial amounts of online student engagement in my courses at all levels and using a classroom response system (clickers) in my high-enrolment ARCHY 101 class. I consult with the UW Instructional Development and Research (CIDR) prior to the start of each quarter to review my syllabus and the effectiveness of assessments. 

I teach classes at every level of the undergraduate and graduate program. My ARCHY 101 ‘Archaeology in Film’ is the highest enrolment archaeology class at UW, with 200 students in Spring 2010. Its high regard amongst students is indicated by the rush for add codes during the registration periods and a combined adjusted median evaluation score of 4.5 (items 1-4) for the Winter 2009 offering. This popularity is in part due to the film content of the class, but more significantly due to the successful articulation of the contemporary social relevant of archaeology to students’ everyday lives. In the Summer of 2010 I will teach ARCHY 270 ‘Field Course’ as an archaeological field school an Australian desert setting with 10 students (seven from UW, three from elsewhere in the US and Canada). This field school will immerse students at the nexus of commercial archaeology and the Australian mining industry, working with BHP, the largest mining company in the world. It is anticipated that this will provide an internally unique combination of academic and professional experience to prepare students for a career in archaeology. 

In the Autumn of 2009 I taught ARCHY 369 ’Special topics: Mainland Southeast Asian Archaeology’ which was a new class for me and the UW, scoring an adjusted median evaluation score of 4.1 (items 1-4) for its initial offering. My main lab class is ARCHY 482, which I inherited from Prof Julie Stein. I have substantially modified this class to reflect my interests and skills as well as my view on the future of the discipline. I also consult with local Cultural Resource Management firms in Seattle to identify skills and abilities that they value in recent graduates that I can cultivate in my class. In modifying this class I have introduced a number of new analytical methods and instruments to the department, such as magnetic susceptibility and phosphorus quantification by spectrophotometry which are now widely used by undergraduate and graduate students. Students’ experience of this geoarchaeology class is very positive as indicated by an adjusted median evaluation score of 4.7 (items 1-4) for the Winter 2010 offering. This geoarchaeology class is also my main source of students that subsequently do independent study (ARCHY 499 or ARCHY 600) with me on more specialized projects and thus a substantial generator of undergraduate research opportunities. Each quarter I advise between two and five undergraduates working on independent study research projects. In addition to my own classes, I have presented guest lectures in other undergraduate courses such as ARCHY 470, ARCHY 205 and ARCH 105. For graduate students I teach ARCHY 570 ‘Archaeology and Explanation’ which is a course in archaeological theory that is especially motivated and informed by philosophy of science work on explanation. The student evaluations showed an adjusted median evaluation score of 4.2 (items 1-4) for the Spring 2009 offering. A most substantial demonstration of the impact of this course is provided by two graduates of the class who were motivated to organize a session on archaeology and explanation at the 2010 Society of American Archaeology Annual Meeting, which included a number of prominent non-UW archaeologists as presenters. 

I currently serve on graduate committees of six students in archaeology and am the first-year adviser of two graduate students (who have not yet formed committees). I am the chair of one graduate committee (Jenn Huff) and the co-chair of another graduate committee (Ashley Dailide, with Don Grayson). I have been an examiner for three general exams (Colby Phillips, Amy Jordan and Emily Peterson) and a committee member for one graduated PhD student (Chris Lockwood, Dec 2009). 

classes

Olympic Dam... MMAP geoarchaeology... Film and explanation... 

\bigskip     

\marginhead{\sffamily service} 

Prepare a table that summarizes your service contributions to the department, college \& or university, and profession (including leadership roles and dates of service). Describe how your service contributions support the mission of your department and of your institution.





Teaching experience



Service contributions

In the two years that I have been employed by the University of Washington, I have worked to serve the department in the following ways:

	2010-current, Website redesign committee
	2010-current, Teaching and technology subcommittee  
2009-current, Coordinator of Archaeology Seminars 
2009-current, Self-Sustaining MA Degree Program in Public Anthropology Planning Committee
2009-current, Coordinator Departmental Honors Program 

Within the archaeology program I also make substantial service contributions such as designing a brochure to advertise the archaeology track, coordination of the comprehensive exams in the graduate program and coordination of undergraduate outreach through the daily maintenance of a Facebook page for the archaeology program. I have made substantial contributions (10-50%) to the text of recent student applications to the Student Technology Fee committee for funds to purchase lab equipment and routinely assist with the administrative and budgeting tasks associated with those proposals. 

Elsewhere on campus I have become involved in various programs:

2010-current, Affiliate faculty Curator, Archaeology Department, Burke Museum (Pending approval)
2010-current, ad-hoc peer-reviewer, UW Royalty Research Fund
2010-current, faculty adviser, South East Asia Center Graduate Student Conference
2010-current, Affiliate Faculty, UW IGERT Program in Evolutionary Modeling (admissions applications reviewer)
2009-current, Affiliate Faculty, UW Center for Statistics and Social Sciences
2009-current, Affiliate Faculty, UW Quaternary Research Center
2009, South East Asia Center MA Program (admissions applications reviewer)

More broadly for the discipline I am active in reviewing scholarly work for publication and applications for funding agencies:

2010-current, Co-editor (with Peter Lape) Bulletin of the Indo-Pacific Prehistory Association
2008-current, Associate Editor, Journal of World Prehistory
2008-current, Ad hoc peer-reviewer for: American Antiquity, Australian Museum Press, C.N.R.S (National Centre for Scientific Research), National University of Singapore Press, ANU E Press, National Science Foundation
2005-current, Ad hoc peer-reviewer for Journal of Archaeological Science, Australian Archaeology, Archaeology in Oceania


\end{document}
