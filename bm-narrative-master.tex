%%% A template to produce a nice-looking Curriculum Vitae.
%%% Originally from Kieran Healy <kjhealy@gmail.com>
%%% Kieran's source is at http://kjhealy.github.com/kjh-vita
%%%
%%% This source code has been adapted by 
%%% Ben Marwick <benmarwick@gmail.com> and is largely
%%% determined by the instructions here:
%%% https://admin.artsci.washington.edu/promotion-and-tenure-documentation#curriculum
%%%
%%% questions: where to put on campus presentation on scholarly and teaching topics?
%%% ------------------------------------------------------------------------
%%% Requirements for this docuemnt that are included
%%%  in (or otherwise easily obtained) a modern tex distribution:
%%% ------------------------------------------------------------------------
%%% xelatex (I'm using MiKTeX & Texmaker)
%%% fontspec.sty
%%% hyperrref.sty
%%% xunicode.sty
%%% color.sty
%%% url.sty
%%% fancyhdr.sty
%%% memoir.cls
%%% fontawesome.sty
%%% gitinfo.sty
%%% 
%%% 
%%% ------------------------------------------------------------------------
%%% Requirements from https://github.com/kjhealy/latex-custom-kjh
%%% ------------------------------------------------------------------------
%%% org-preamble-xelatex.sty
%%% memoir-article-styles.sty
%%%
%%% 
%%% ------------------------------------------------------------------------
%%% Note
%%%------------------------------------------------------------------------
%%% Because this is a hand-tweaked file, be on the look out for \medksip, 
%%% \bigskip and \newpage commands here and there, which are used to balance
%%% the layout or avoid widows & orphans, etc. You should of course add or 
%%% remove these as needed.
%%%------------------------------------------------------------------------

\documentclass[11pt,article,oneside]{memoir}   
\usepackage{microtype}
\usepackage{org-preamble-xelatex} 
\usepackage{fontawesome,url}
\usepackage{setspace}
\usepackage[footinfo]{gitinfo}



%%%------------------------------------------------------------------------
%%% Metadata
%%%------------------------------------------------------------------------

%% Change as needed.
\def\myauthor{Ben Marwick}
\def\mytitle{Vita}
\def\mycopyright{\myauthor}
\def\mykeywords{}
\def\mybibliostyle{plain}
\def\mybibliocommand{}
\def\mysubtitle{}
\def\myaffiliation{University of Washington}
\def\myaddress{Anthropology Department}
\def\myemail{bmarwick@uw.edu}
\def\myweb{http://faculty.washington.edu/bmarwick/ }
\def\myphone{(206) 552-9450}
\def\myfax{(206) 543-3285}
\def\mytwitter{@benmarwick}
\def\myversion{}
\def\myrevision{}


\def\myaffiliation{University of Washington}
\def\myauthor{Ben Marwick}
\date{} % not used (revision control instead)
\def\mykeywords{}


%%%------------------------------------------------------------------------
%%% Document
%%%------------------------------------------------------------------------
\begin{document}

%% Choose fonts for use with xelatex
%% Using free fonts with big glyph sets for maximum flexibility  
%% http://www.linuxlibertine.org/index.php?id=2&L=1
%% http://font.ubuntu.com/

\setromanfont[Mapping={tex-text}, 
	Numbers={OldStyle},
	Ligatures={Common}]{Linux Libertine}
\setsansfont[Mapping=tex-text,
	Ligatures={Common}, 
	Colour=AA0000]{Linux Biolinum}
\setmonofont[Mapping=tex-text,Scale=0.72]{Ubuntu} 

\newfontface\scheader[SmallCapsFont={Linux Libertine},SmallCapsFeatures={Letters=SmallCaps}]{Linux Libertine}

\newfontface\addressblock[Mapping={tex-text}, 
	Numbers={OldStyle},
	Ligatures={Common}]{Linux Libertine}


%%%------------------------------------------------------------------------
%%% Local commands
%%%------------------------------------------------------------------------

%% Marginal header
%% Note: as the document goes on you may need to introduce a (gradually increasing)
%% \vspace element to keep the marginal header pleasingly aligned with the first 
%% item in the body text. Like this: \marginhead{{\vskip 0.4em}Grants}, or 
%% \marginhead{{\vskip 0.8em}Service}. Experiment as needed.
\newcommand{\marginhead}[1]{\marginpar{\textsf{{\footnotesize\vspace{-1em}\flushright #1}}}}


%% [optional] custom ampersand (font consistent with the one chosen above)
\newcommand{\amper}{{\fontspec[Scale=.95,Colour=AA0000]{Linux Libertine}\selectfont\&\,}}

%% No bullets on labels
\renewcommand{\labelitemi}{~} 

%% Custom hanging indent for vita items
\def\ind{\hangindent=1 true cm\hangafter=1 \noindent}
%\def\ind{\hangindent=18pt\hangafter=1 \noindent}
\def\labelitemi{~}
\renewcommand{\labelitemii}{~}

%%%------------------------------------------------------------------------
%%% Page layout
%%%------------------------------------------------------------------------

% These lines will insert git revision info in the footer, using the gitinfo
% package---see docs for gitinfo package for details. Comment out this line
% if you're not using git.
\pagestyle{kjh}
\thispagestyle{kjhgit}

%%%------------------------------------------------------------------------
%%% Address and contact block
%%%------------------------------------------------------------------------
\begin{minipage}[t]{2.95in}
 \flushright {\footnotesize 
 \href{http://depts.washington.edu/anthweb/}{Department of Anthropology} \\ Box 353100 \\ University of Washington  \\ \vspace{-0.05in} Seattle \textsc{wa} 98195-3100}  
  
\end{minipage}
\hfill     
%\begin{minipage}[t]{0.0in}
% dummy (needed here)
%\end{minipage}
\hfill
\begin{minipage}[t]{1.7in}
  \flushright \footnotesize  \addressblock \myphone \, \faPhone \\ 
  {\scriptsize  \texttt{\href{http://twitter.com/benmarwick}{\mytwitter}} \, \faTwitter }  \\ 
  {\scriptsize  \texttt{\href{mailto:\myemail}{\myemail}} \, \faEnvelope} \\
  {\scriptsize  \texttt{\href{\myweb}{\myweb}} \, \faGlobe}
\end{minipage}

\medskip

%% Name 
\noindent{\LARGE\scheader \textsc{personal statement}}
\reversemarginpar

\bigskip       

% Personal statement for tenure
% http://admin.artsci.washington.edu/promotion-and-tenure-documentation#candidates

% The statement should not be longer than three to five pages.
%
% In the discussion of research or scholarship, a short essay is more effective than an annotated list of works. The candidate's research contributions might be described in the broader context of the discipline as a whole, explaining how his or her research agenda fits into the discipline and then how particular scholarly or creative contributions fit into this agenda. The essay should also include the candidate's statement of future directions and how these connect to previous and current work, in order to give a sense of the trajectory of the work.
%
% The personal statement should contain as well a discussion of the candidate's teaching experience, with an overview of the candidate's goals, a review of successes and failures, reflections on these experiences, and thoughts of what lies ahead. Lastly, the candidate should describe any significant service contributions.

% ideas...
% http://www.insidehighered.com/advice/2010/11/10/narrative
% http://www.slideshare.net/UO-AcademicAffairs/writing-a-tenure-statement-2011

\bigskip       

\marginhead{\sffamily introduction}

I am an archaeologist interested in human behavior, technology and ecology. My approach to these themes is motivated by models and methods from the evolutionary sciences. I seek to adapt these methods and models to better understand the human past. My main research activities include investigating questions of technological variation and ecological adaptation, cultural change and cultural transmission. I have a deep interest in mainland Southeast Asia and Australia, and fieldwork in these locations supplies the empirical content of my research. Fundamental to these explorations are my concerns for the social relevance and disciplinary integrity for archaeology.

I began my research career investigating questions about Australian Aboriginal hunter-gatherer adaptation and mobility during the Late Pleistocene and Holocene periods  {\href{http://dx.doi.org/10.6084/m9.figshare.765251}{(Marwick 2002b)}}, ({\href{http://hdl.handle.net/1885/42085}{ 2002a}}, {\href{http://faculty.washington.edu/bmarwick/PDFs/Marwick_2005_Marillana_A.pdf}{ 2005b}}). I analyzed stone artefact technology and developed new geoarchaeological methods for identifying variation in intensity and duration of site occupation for the inland Pilbara region of Western Australia. These methods were developed to address the general problem of robustly explaining past human behavior using archaeological deposits that have low densities of artefacts. Motivated by the potential of stone artefact and geoarchaeological analysis to investigate archaeological problems that are intractable with conventional archaeological approaches, I extended my empirical research into Southeast Asia. My dissertation research {\href{http://dx.doi.org/10.6084/m9.figshare.765252}{(Marwick 2008}} into the relationship between stone artefact technological variation and local ecological conditions in Thailand was a driven by the question of how the ecological conditions of the seasonal tropics related to decisions people made to change their stone artefact technology or change their settlement locations in the tropics. 

My research draws on three motives for working in mainland Southeast Asia. I am first motivated to understand the histories and adaptations of Middle and Late Pleistocene hominins in Southeast Asia. This is an extension of my Australian research, as these early colonists of Southeast Asia  were probably ancestral to the first human occupants of Australia. My second motive is to understand how hunter-gatherers in Southeast Asia changed during the transition to domestication, a transition that did not occur in prehistoric times in Australia but has profoundly shaped modern Southeast Asian society. Finally, I am energized by the dual practical challenges of working in region where knowledge about human prehistory is sparse and contributing to the development of local institutions and individuals that research, teach and promote the social relevance of archaeology. 

In what follows, I will describe my research trajectory since arriving at the University of Washington in 2008 and explain how the characteristics that distinguish my research – namely, its philosophical engagement with scientific practice and its focus on improving judgment through social change – contribute to my teaching and service.

\bigskip     

\marginhead{\sffamily research \newline trajectory}

Emerging from my dissertation research has been a collaboration with a colleague from the Earth Sciences to construct a local palaeoclimate proxy from oxygen isotope values obtained from shell fish recovered from the same archaeological deposit as the stone artefacts. We published this study in \textit{Quaternary Science Reviews}  {\href{http://faculty.washington.edu/bmarwick/PDFs/Marwick_and_Gagan_2011_QSR.pdf}{(Marwick and Gagan 2011)}}. This proxy is the first oxygen isotope curve for mainland Southeast Asia which so far has been dependent on discontinuous pollen sequences for palaeoclimate data. Our oxygen isotope curve provides the first evidence of a major change in conditions from the Pleistocene to the Holocene, a finding consistent with evidence from similar records in China. A second extension from my dissertation has been a paper in the \textit{Journal of Anthropological Archaeology} describing a modified behavioural ecological model {\href{http://www.academia.edu/4845513/Multiple_Optima_in_Hoabinhian_flaked_stone_artefact_palaeoeconomics_and_palaeoecology_at_two_archaeological_sites_in_Northwest_Thailand}{(Marwick 2013)}} that resolves the apparent inconsistencies in the data from Thailand by proposed a novel general model of multiple optima. 

With this early work situated in an inland upland setting, I was curious to evaluate the usefulness of my model in different tropical landscapes. In 2011 I initiated a project with Thai collaborators to explore hunter-gatherer adaptations in coastal environments with an excavation and field school in Krabi, southern Thailand. 


but I have been involved in extensive fieldwork and data analysis for the Olympic Dam Project. This has resulted in my co-authorship of two scholarly conference presentations and three unpublished consultant reports. I wrote both of the conference presentations and contributed about 30\% of the text and analysis of the reports. A manuscript is in preparation for the Journal of Field Archaeology for results of the first three years for the Olympic Dam work (I am first author on this manuscript). I will be directing an archaeological field school for UW students in the Summer Quarter of 2010 as part of the Olympic Dam Expansion Project. The field school will allow me to continue my work on Australian stone artefact technology and adaptation with a new focus on arid environments and analysis of site distribution patterns. I expect this will result in one or two conference presentations at the Australian Archaeological Association Annual Conference and the Society of American Archaeology in 2011. I have also recently collaborated on project with four Australian colleagues working on Late Holocene shell middens in northern Australia. My contribution was an analysis of oxygen isotope ratios in shells sampled from excavated conducted by my colleagues. I am second author on a manuscript currently under review with the Journal of Archaeological Science (I contributed about 50\% of the text). 


\bigskip     

\marginhead{\sffamily teaching}

Describe your philosophy regarding the teaching and training of the next generation of scientists and, if appropriate non-scientists (for example, general education students or future K-12 teachers). Prepare a table that summarizes your teaching activity semester by semester (including course number, course title, number of students, and course evaluation information); acknowledge if the course is co-taught.

\bigskip     

\marginhead{\sffamily service} 

Prepare a table that summarizes your service contributions to the department, college \& or university, and profession (including leadership roles and dates of service). Describe how your service contributions support the mission of your department and of your institution.




Hominin colonization and adaptation in mainland Southeast Asia

My dissertation research reached back to about 40,000 years, and whilst engaged in that work I became interested in the origins of the stone artefact technology of that Late Pleistocene period, and the origins of the people that made those tools. This interest was also motivated by my previous work on Pleistocene stone artefact assemblages in Australia which stimulated questions about the relationship of early Australian technologies to antecedent technologies in Southeast Asia. At the core of this interest is the question of the relative importance of ancestry versus adaptation of stone artefact technologies, which is essentially the evolutionary biology question of homology versus analogy. My interests in these questions in mainland Southeast Asia were further stimulated by the discovery of Homo floresiensis in island Southeast Asia in 2004. Evaluating the uniqueness of those finds and understanding their place in the evolutionary sequence in Southeast Asia is a motivating concern for me.

To date I have published a review article in Quaternary International where I present a colonization model that I intend to test with fieldwork. In 2008 I conducted survey of three locations in northern Thailand with potential for Middle Pleistocene archaeological deposits. I also am second author on a paper under review with Journal of World Prehistory discussing the Middle Pleistocene East and Southeast Asian archaeological record (my contribution to the paper is about 10% of the text and extensive edits). I drafted and have recently concluded MOUs with Mahasarakham University and Chiang Mai University (both in Thailand) where my collaborators work. These MOUs are a prerequisite for an application in preparation for the Leakey Foundation to fund a field season of excavation at one of the previously surveyed locations. The goal is to recover securely dated stone artefacts from before 40,000 years ago and examine their relationship to later technologies in Southeast Asia and Australia. 

Transitions to agriculture in mainland Southeast Asia

Just beyond the recent end of the chronological spectrum of my dissertation research is the transformation of hunter-gatherers to farmers in mainland Southeast Asia. Basic questions of timing, history and process remain highly contested for this transformation in this region and I am motivated by the potential of geoarchaeological techniques to extract data from an otherwise sparse and challenging archaeological record. In 2005 I joined the Middle Mekong Archaeological Project (MMAP), directed by Joyce White of the University of Pennsylvania Museum. The goals of this project aligned exactly with my interests and it have given me an excellent opportunity to develop collegial relationships with key officials in the Lao PDR. I have discussed this process of collaboration in a recent article for the SAA Archaeological Record. I conceived and wrote this entire article, with my co-authors (the directors of the MMAP project) contributing minor edits. 

The substantive results of MMAP’s fieldwork have so far not been successful in locating archaeological deposits relevant to the topic of transitions to agriculture. Instead the most reliable data collected by MMAP has been on later populations from the Iron Age. I am a co-author (fourth author, contributing about 10% of the text and extensive edits) on a brief note on results to date published in Antiquity. In the summer of 2009 I directed an MMAP excavation in northern Laos with the assistance of UW graduate student Andy Cowan. Although the MMAP agreement with the Laos prevents exporting of any archaeological materials to the US, I was able to procure sediment samples and ceramic samples for further analysis at UW. These samples have given intensive research opportunities to a number of graduate and undergraduate students under my supervision. For example, students in my ARCHY 482 Geoarchaeology class have analysed the sediments, my graduate students Anna Cohen and Seungki Kwak conducted an organic geochemical analysis of residues in the ceramics, one undergraduate has done a detailed elemental study of the clays in the ceramics using XRF, other undergraduates have studies pollen and clay minerals in the sediments. Andy Cowan also dated the ceramics in Prof Feathers’ luminescence dating lab. Some of this work was presented by me, with students as co-authors, at the 2009 Indo-Pacific Prehistory Association Congress in Hanoi. I am currently preparing a report for the Lao PDR Ministry of Information and Culture which I intend to adapt for publication in Asian Perspectives. My undergraduate students who have worked on this material will give posters in the 2010 UW Undergraduate Research Symposium and the graduate students will prepare a poster for the 2011 Society of American Archaeology meeting.  

In September 2010 I will begin a ten month post-doctoral fellowship funding by the Luce Foundation to continue working in Laos. My intention with this fellowship is to refocus on early Holocene hunter-gatherers and the transition to agriculture with my MMAP colleagues, but the outcomes of this work will be determined by the interests and objectives of my Lao colleagues, who may be more likely to encourage research into later periods, given the success of my recent analyses of the 2009 excavation. Funding for fieldwork and analytical work has so far come from my faculty startup budget and I intend to apply the Wenner-Gren Foundation for fieldwork funds to support my post-doctoral fellowship. In the summer of 2011 I plan to conduct an archaeological field school for UW students in Laos as part of Peter Lape’s Luce-funded field schools in Southeast Asia. 

Cultural transmission in Iron Age communities in mainland Southeast Asia

Whilst working on prehistoric archaeology in mainland Southeast Asia I developed an interest in the potential of more recent material culture to yield insights into historical and political processes by analyzing fine art objects with techniques normally used on biological populations. In particular, I have been pursuing attribute-based phylogenies and morphometric analyses of bronze Buddha statues from Laos and Thailand to better understand the flow of influence between different production centers. With financial support from the UW Center for Statistics and Social Sciences I have conducted one brief field season in the Luang Prabang National Museum with my collaborator, Vanpheng Keopanna, the museum’s director. I am currently engaged in a desktop study and data analysis and am scheduled to present some results in a conference paper at the European Association of Southeast Asian Archaeology conference in September 2010 and publish in the edited volume that follows that conference. I plan to apply for funds in 2011 to pursue this work further. 

Social relevance and disciplinary integrity 

These research interests have evolved out of my teaching activities UW. First is my ARCHY 101 ‘Archaeology in Film’ class, where I use popular blockbuster films to demonstrate the everyday relevance of archaeological thinking, or thinking about the long-term, about to meaning of objects in our lives and about how notions of the past define our daily behaviors. I have presented a novel exploration of some of these concepts in a peer-reviewed article accepted for publication in a forthcoming (2010) issue of World Archaeology dedicated to the theme of ‘Archaeology and Contemporary Society’. In 2009 I gave an invited talk at the Burke Museum discussing the contemporary relevant of fictional films about ice age caveman films. I have also discussed cross-cultural and international collaboration in archaeology in a paper for the SAA Archaeological Record, in a special issue of papers on international collaboration that I co-edited with Christian Peterson. 

Second is my ARCHY 570 ‘Archaeology and Explanation’ class, which conducts a parallel investigation of approaches to explanation by philosophers of science and by archaeologists. The idea for this class grew out of a concern that archaeologists could be better informed about some of the basic intellectual activities that they do when making sense of the human past. My students from this class organized a session on archaeology in explanation at the 2010 Society of American Archaeology which included three senior non-UW archaeologists noted for their past philosophical contributions. I contributed a paper advocating model-based explanations, which has already been requested several times in draft form and I am currently preparing for publication.

I am yet to develop a long term strategy specifically to pursue these two research interests. Currently my plan is to incorporate work on explanation and modeling into my empirical work and pursue my film and social relevance work as opportunities arise. 

Teaching experience

My teaching philosophy and methods are informed by evidence-based findings about US college pedagogy and benefit from the same kind of study of peer-reviewed literature as my archaeological work. My general approach is to stimulate critical thinking about the human past and is motivated by seven principles distilled by Chickering and Gamson in their 1987 a meta-analysis of 50 years of research on good teaching principles. These include: encouraging contact between students and faculty, developing reciprocity and cooperation among students, encouraging active learning, giving prompt feedback, emphasizing time on task, communicating high expectations, and respecting diverse talents and ways of learning. These principles are the foundation of my syllabus design, course scheduling, assignments and the in-person engagement I have with my students. At the level of planning individual lessons, I organize my class time according to the three factors identified by the National Research Council report “How People Learn” (2000) as critical to learning: addressing student’s preconceptions of the discipline, helping students develop a rich conceptual framework into which to place factual knowledge, and enhancing a student’s ability to monitor their learning. I also routinely consult peer-reviewed literature on data visualization, cognitive load theory and student information literacy to continuously improve my engagement with students in and out of the classroom. I am an intensive user of technology in my teaching, employing substantial amounts of online student engagement in my courses at all levels and using a classroom response system (clickers) in my high-enrolment ARCHY 101 class. I consult with the UW Instructional Development and Research (CIDR) prior to the start of each quarter to review my syllabus and the effectiveness of assessments. 

I teach classes at every level of the undergraduate and graduate program. My ARCHY 101 ‘Archaeology in Film’ is the highest enrolment archaeology class at UW, with 200 students in Spring 2010. Its high regard amongst students is indicated by the rush for add codes during the registration periods and a combined adjusted median evaluation score of 4.5 (items 1-4) for the Winter 2009 offering. This popularity is in part due to the film content of the class, but more significantly due to the successful articulation of the contemporary social relevant of archaeology to students’ everyday lives. In the Summer of 2010 I will teach ARCHY 270 ‘Field Course’ as an archaeological field school an Australian desert setting with 10 students (seven from UW, three from elsewhere in the US and Canada). This field school will immerse students at the nexus of commercial archaeology and the Australian mining industry, working with BHP, the largest mining company in the world. It is anticipated that this will provide an internally unique combination of academic and professional experience to prepare students for a career in archaeology. 

In the Autumn of 2009 I taught ARCHY 369 ’Special topics: Mainland Southeast Asian Archaeology’ which was a new class for me and the UW, scoring an adjusted median evaluation score of 4.1 (items 1-4) for its initial offering. My main lab class is ARCHY 482, which I inherited from Prof Julie Stein. I have substantially modified this class to reflect my interests and skills as well as my view on the future of the discipline. I also consult with local Cultural Resource Management firms in Seattle to identify skills and abilities that they value in recent graduates that I can cultivate in my class. In modifying this class I have introduced a number of new analytical methods and instruments to the department, such as magnetic susceptibility and phosphorus quantification by spectrophotometry which are now widely used by undergraduate and graduate students. Students’ experience of this geoarchaeology class is very positive as indicated by an adjusted median evaluation score of 4.7 (items 1-4) for the Winter 2010 offering. This geoarchaeology class is also my main source of students that subsequently do independent study (ARCHY 499 or ARCHY 600) with me on more specialized projects and thus a substantial generator of undergraduate research opportunities. Each quarter I advise between two and five undergraduates working on independent study research projects. In addition to my own classes, I have presented guest lectures in other undergraduate courses such as ARCHY 470, ARCHY 205 and ARCH 105. For graduate students I teach ARCHY 570 ‘Archaeology and Explanation’ which is a course in archaeological theory that is especially motivated and informed by philosophy of science work on explanation. The student evaluations showed an adjusted median evaluation score of 4.2 (items 1-4) for the Spring 2009 offering. A most substantial demonstration of the impact of this course is provided by two graduates of the class who were motivated to organize a session on archaeology and explanation at the 2010 Society of American Archaeology Annual Meeting, which included a number of prominent non-UW archaeologists as presenters. 

I currently serve on graduate committees of six students in archaeology and am the first-year adviser of two graduate students (who have not yet formed committees). I am the chair of one graduate committee (Jenn Huff) and the co-chair of another graduate committee (Ashley Dailide, with Don Grayson). I have been an examiner for three general exams (Colby Phillips, Amy Jordan and Emily Peterson) and a committee member for one graduated PhD student (Chris Lockwood, Dec 2009). 

Service contributions

In the two years that I have been employed by the University of Washington, I have worked to serve the department in the following ways:

	2010-current, Website redesign committee
	2010-current, Teaching and technology subcommittee  
2009-current, Coordinator of Archaeology Seminars 
2009-current, Self-Sustaining MA Degree Program in Public Anthropology Planning Committee
2009-current, Coordinator Departmental Honors Program 

Within the archaeology program I also make substantial service contributions such as designing a brochure to advertise the archaeology track, coordination of the comprehensive exams in the graduate program and coordination of undergraduate outreach through the daily maintenance of a Facebook page for the archaeology program. I have made substantial contributions (10-50%) to the text of recent student applications to the Student Technology Fee committee for funds to purchase lab equipment and routinely assist with the administrative and budgeting tasks associated with those proposals. 

Elsewhere on campus I have become involved in various programs:

2010-current, Affiliate faculty Curator, Archaeology Department, Burke Museum (Pending approval)
2010-current, ad-hoc peer-reviewer, UW Royalty Research Fund
2010-current, faculty adviser, South East Asia Center Graduate Student Conference
2010-current, Affiliate Faculty, UW IGERT Program in Evolutionary Modeling (admissions applications reviewer)
2009-current, Affiliate Faculty, UW Center for Statistics and Social Sciences
2009-current, Affiliate Faculty, UW Quaternary Research Center
2009, South East Asia Center MA Program (admissions applications reviewer)

More broadly for the discipline I am active in reviewing scholarly work for publication and applications for funding agencies:

2010-current, Co-editor (with Peter Lape) Bulletin of the Indo-Pacific Prehistory Association
2008-current, Associate Editor, Journal of World Prehistory
2008-current, Ad hoc peer-reviewer for: American Antiquity, Australian Museum Press, C.N.R.S (National Centre for Scientific Research), National University of Singapore Press, ANU E Press, National Science Foundation
2005-current, Ad hoc peer-reviewer for Journal of Archaeological Science, Australian Archaeology, Archaeology in Oceania


\end{document}
