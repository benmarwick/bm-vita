%%% A template to produce a nice-looking Curriculum Vitae.
%%% Originally from Kieran Healy <kjhealy@gmail.com>
%%% Kieran's source is at http://kjhealy.github.com/kjh-vita
%%%
%%% This source code has been adapted by 
%%% Ben Marwick <benmarwick@gmail.com> and is largely
%%% determined by the instructions here:
%%% \{https://admin.artsci.washington.edu/promotion-and-tenure-documentation#curriculum}
%%%
%%% ------------------------------------------------------------------------
%%% Requirements for this docuemnt that are included
%%%  in (or otherwise easily obtained) a modern tex distribution:
%%% ------------------------------------------------------------------------
%%% xelatex (I'm using MiKTeX & Texmaker)
%%% fontspec.sty
%%% hyperrref.sty
%%% xunicode.sty
%%% color.sty
%%% url.sty
%%% fancyhdr.sty
%%% memoir.cls
%%% fontawesome.sty
%%% gitinfo.sty
%%% 
%%% 
%%% ------------------------------------------------------------------------
%%% Requirements from https://github.com/kjhealy/latex-custom-kjh
%%% ------------------------------------------------------------------------
%%% org-preamble-xelatex.sty
%%% memoir-article-styles.sty
%%%
%%% 
%%% ------------------------------------------------------------------------
%%% Note
%%%------------------------------------------------------------------------
%%% Because this is a hand-tweaked file, be on the look out for \medksip, 
%%% \bigskip and \newpage commands here and there, which are used to balance
%%% the layout or avoid widows & orphans, etc. You should of course add or 
%%% remove these as needed.
%%%------------------------------------------------------------------------

\documentclass[11pt,article,oneside,oldfontcommands]{memoir}   
\PassOptionsToPackage{table}{xcolor}
\usepackage{microtype}
\usepackage{org-preamble-xelatex} 
\usepackage{fontawesome,url}
\usepackage{setspace}
\usepackage[footinfo]{gitinfo}

% for B&W lines in table
\usepackage{booktabs}
\usepackage[table]{xcolor}
\definecolor{light-gray}{gray}{0.95}
% use this to go from CSV to latex: http://truben.no/latex/table/


%%%------------------------------------------------------------------------
%%% Metadata
%%%------------------------------------------------------------------------

%% Change as needed.
\def\myauthor{Ben Marwick}
\def\mytitle{Vita}
\def\mycopyright{\myauthor}
\def\mykeywords{}
\def\mybibliostyle{plain}
\def\mybibliocommand{}
\def\mysubtitle{}
\def\myaffiliation{University of Washington}
\def\myaddress{Anthropology Department}
\def\myemail{bmarwick@uw.edu}
\def\myweb{http://faculty.washington.edu/bmarwick/ }
\def\myphone{(206) 552-9450}
\def\myfax{(206) 543-3285}
\def\mytwitter{@benmarwick}
\def\myversion{}
\def\myrevision{}


\def\myaffiliation{University of Washington}
\def\myauthor{Ben Marwick}
\date{} % not used (revision control instead)
\def\mykeywords{}


%%%------------------------------------------------------------------------
%%% Document
%%%------------------------------------------------------------------------

\begin{document}

%% Choose fonts for use with xelatex
%% Using free fonts with big glyph sets for maximum flexibility  
%% http://www.linuxlibertine.org/index.php?id=2&L=1
%% http://font.ubuntu.com/

\setromanfont[Mapping={tex-text}, 
	Numbers={OldStyle},
	Ligatures={Common}]{Linux Libertine}
\setsansfont[Mapping=tex-text,
	Ligatures={Common}, 
	Colour=AA0000]{Linux Biolinum}
\setmonofont[Mapping=tex-text,Scale=0.72]{Ubuntu} 

\newfontface\scheader[SmallCapsFont={Linux Libertine},SmallCapsFeatures={Letters=SmallCaps}]{Linux Libertine}

\newfontface\addressblock[Mapping={tex-text}, 
	Numbers={OldStyle},
	Ligatures={Common}]{Linux Libertine}


%%%------------------------------------------------------------------------
%%% Local commands
%%%------------------------------------------------------------------------

%% Marginal header
%% Note: as the document goes on you may need to introduce a (gradually increasing)
%% \vspace element to keep the marginal header pleasingly aligned with the first 
%% item in the body text. Like this: \marginhead{{\vskip 0.4em}Grants}, or 
%% \marginhead{{\vskip 0.8em}Service}. Experiment as needed.
\newcommand{\marginhead}[1]{\marginpar{\textsf{{\footnotesize\vspace{-1em}\flushright #1}}}}


%% [optional] custom ampersand (font consistent with the one chosen above)
\newcommand{\amper}{{\fontspec[Scale=.95,Colour=AA0000]{Linux Libertine}\selectfont\&\,}}

%% No bullets on labels
\renewcommand{\labelitemi}{~} 

%% Custom hanging indent for vita items
\def\ind{\hangindent=1 true cm\hangafter=1 \noindent}
%\def\ind{\hangindent=18pt\hangafter=1 \noindent}
\def\labelitemi{~}
\renewcommand{\labelitemii}{~}

%% table row shading


%%%------------------------------------------------------------------------
%%% Page layout
%%%------------------------------------------------------------------------

% These lines will insert git revision info in the footer, using the gitinfo
% package---see docs for gitinfo package for details. Comment out this line
% if you're not using git.
\pagestyle{kjh}
\thispagestyle{kjhgit}

%%%------------------------------------------------------------------------
%%% Address and contact block
%%%------------------------------------------------------------------------
%\begin{minipage}[t]{2.95in}
% \flushright {\footnotesize 
% \href{http://depts.washington.edu/anthweb/}{Department of Anthropology} \\ Box 353100 \\ University of Washington  \\ \vspace{-0.05in} Seattle \textsc{wa} 98195-3100}  
%  
%\end{minipage}
%\hfill     
%%\begin{minipage}[t]{0.0in}
%% dummy (needed here)
%%\end{minipage}
%\hfill
%\begin{minipage}[t]{1.7in}
%  \flushright \footnotesize  \addressblock \myphone \, \faPhone \\ 
%  {\scriptsize  \texttt{\href{http://twitter.com/benmarwick}{\mytwitter}} \, \faTwitter }  \\ 
%  {\scriptsize  \texttt{\href{mailto:\myemail}{\myemail}} \, \faEnvelope} \\
%  {\scriptsize  \texttt{\href{\myweb}{\myweb}} \, \faGlobe}
%\end{minipage}

% http://admin.artsci.washington.edu/promotion-and-tenure-documentation
%a list of all courses taught at the UW, with dates
%a list of graduate students supervised, with student name, thesis topic, degree, dates, and the faculty member's committee role (chair or member)
%student assessment of teaching
%collegial assessment of teaching
%an analysis of the complete teaching record by the chair and, if possible, a departmental committee

\medskip

%% Name 
%% Name 
\noindent{\LARGE\scheader \textsc{teaching effectiveness}}
\reversemarginpar

\bigskip   

\marginhead{\sffamily {{\vskip 2.5em} courses \newline taught at \newline UW}}

\def\mainfont{Gill Sans MT}
\font\bodyfont="\mainfont:mapping=tex-text;+onum" at 8bp \let\tenrm\bodyfont
\font\boldfont="\mainfont/B" at 8bp \let\tenbf\boldfont
\bodyfont

\begin{center}
\rowcolors{1}{white}{light-gray}
\begin{tabular}{lrrr} 
   Course code& Course name& Number of students& Offered\\
  \midrule
    ARCHY 109    & Archaeology in Film            & 25, 151, 189, 142           & SU08, WI09, SP12, WI14\\ 
    ARCHY 270    & Field School                         & 10, 10                            & SU10, SU12\\
    ARCHY 309    & Mainland Southeast Asian Archaeology & 20, 12       & AU09, SP14\\
    ARCHY 319    & Australian Archaeology      & 16                                  & WI13\\
    ANTH 399     & Undergraduate Honors Seminar         & 20                  & SP08\\
    ARCHY 482    & Geoarchaeology                  & 15, 15, 11, 16                 & AU08, WI10, WI12, AU13\\
    ARCHY 509    & Archaeology and Explanation   & 8, 3                        & SP08, SP13\\
    \bottomrule
\end{tabular}
\end{center}

\marginhead{\sffamily {{\vskip 0.3em} graduate \newline students \newline supervised}}

don't forget off-campus ones!


\marginhead{\sffamily {{\vskip 0.3em} student\newline assessment \newline of teaching}}



\marginhead{\sffamily {{\vskip 0.3em} collegial \newline assessment \newline of teaching}}


\end{document}
