%%% A template to produce a nice-looking Curriculum Vitae.
%%% Originally from Kieran Healy <kjhealy@gmail.com>
%%% Kieran's source is at http://kjhealy.github.com/kjh-vita
%%%
%%% This source code has been adapted by 
%%% Ben Marwick <benmarwick@gmail.com> and is largely
%%% determined by the instructions here:
%%% https://admin.artsci.washington.edu/promotion-and-tenure-documentation#curriculum
%%%
%%% questions: where to put on campus presentation on scholarly and teaching topics?
%%% ------------------------------------------------------------------------
%%% Requirements for this docuemnt that are included
%%%  in (or otherwise easily obtained) a modern tex distribution:
%%% ------------------------------------------------------------------------
%%% xelatex (I'm using MiKTeX & Texmaker)
%%% fontspec.sty
%%% hyperrref.sty
%%% xunicode.sty
%%% color.sty
%%% url.sty
%%% fancyhdr.sty
%%% memoir.cls
%%% fontawesome.sty
%%% gitinfo.sty
%%% 
%%% 
%%% ------------------------------------------------------------------------
%%% Requirements from https://github.com/kjhealy/latex-custom-kjh
%%% ------------------------------------------------------------------------
%%% org-preamble-xelatex.sty
%%% memoir-article-styles.sty
%%%
%%% 
%%% ------------------------------------------------------------------------
%%% Note
%%%------------------------------------------------------------------------
%%% Because this is a hand-tweaked file, be on the look out for \medksip, 
%%% \bigskip and \newpage commands here and there, which are used to balance
%%% the layout or avoid widows & orphans, etc. You should of course add or 
%%% remove these as needed.
%%%------------------------------------------------------------------------

\documentclass[11pt,article,oneside]{memoir}   
\usepackage{microtype}
\usepackage{org-preamble-xelatex} 
\usepackage{fontawesome,url}
\usepackage{setspace}
\usepackage[footinfo]{gitinfo}

% get gitinfo setup file from http://www.ctan.org/tex-archive/macros/latex/contrib/gitinfo
% and follow instructions in manual before first commit
% http://mirrors.ibiblio.org/CTAN/macros/latex/contrib/gitinfo/gitinfo.pdf



%%%------------------------------------------------------------------------
%%% Metadata
%%%------------------------------------------------------------------------

%% Change as needed.
\def\myauthor{Ben Marwick}
\def\mytitle{Vita}
\def\mycopyright{\myauthor}
\def\mykeywords{}
\def\mybibliostyle{plain}
\def\mybibliocommand{}
\def\mysubtitle{}
\def\myaffiliation{University of Washington}
\def\myaddress{Anthropology Department}
\def\myemail{bmarwick@uw.edu}
\def\myweb{http://faculty.washington.edu/bmarwick/ }
\def\myphone{(206) 552-9450}
\def\myfax{(206) 543-3285}
\def\mytwitter{@benmarwick}
\def\myversion{}
\def\myrevision{}


\def\myaffiliation{University of Washington}
\def\myauthor{Ben Marwick}
\date{} % not used (revision control instead)
\def\mykeywords{}


%%%------------------------------------------------------------------------
%%% Document
%%%------------------------------------------------------------------------
\begin{document}

%% Choose fonts for use with xelatex
%% Using free fonts with big glyph sets for maximum flexibility  
%% http://www.linuxlibertine.org/index.php?id=2&L=1
%% http://font.ubuntu.com/

\setromanfont[Mapping={tex-text}, 
	Numbers={OldStyle},
	Ligatures={Common}]{Linux Libertine}
\setsansfont[Mapping=tex-text,
	Ligatures={Common}, 
	Colour=AA0000]{Linux Biolinum}
\setmonofont[Mapping=tex-text,Scale=0.72]{Ubuntu} 

\newfontface\scheader[SmallCapsFont={Linux Libertine},SmallCapsFeatures={Letters=SmallCaps}]{Linux Libertine}

\newfontface\addressblock[Mapping={tex-text}, 
	Numbers={OldStyle},
	Ligatures={Common}]{Linux Libertine}


%%%------------------------------------------------------------------------
%%% Local commands
%%%------------------------------------------------------------------------

%% Marginal header
%% Note: as the document goes on you may need to introduce a (gradually increasing)
%% \vspace element to keep the marginal header pleasingly aligned with the first 
%% item in the body text. Like this: \marginhead{{\vskip 0.4em}Grants}, or 
%% \marginhead{{\vskip 0.8em}Service}. Experiment as needed.
\newcommand{\marginhead}[1]{\marginpar{\textsf{{\footnotesize\vspace{-1em}\flushright #1}}}}


%% [optional] custom ampersand (font consistent with the one chosen above)
\newcommand{\amper}{{\fontspec[Scale=.95,Colour=AA0000]{Linux Libertine}\selectfont\&\,}}

%% No bullets on labels
\renewcommand{\labelitemi}{~} 

%% Custom hanging indent for vita items
\def\ind{\hangindent=1 true cm\hangafter=1 \noindent}
%\def\ind{\hangindent=18pt\hangafter=1 \noindent}
\def\labelitemi{~}
\renewcommand{\labelitemii}{~}

%%%------------------------------------------------------------------------
%%% Page layout
%%%------------------------------------------------------------------------

% These lines will insert git revision info in the footer, using the gitinfo
% package---see docs for gitinfo package for details. Comment out this line
% if you're not using git.
\pagestyle{kjh}
\thispagestyle{kjhgit}

%%%------------------------------------------------------------------------
%%% Address and contact block
%%%------------------------------------------------------------------------
\begin{minipage}[t]{2.95in}
 \flushright {\footnotesize 
 \href{http://depts.washington.edu/anthweb/}{Department of Anthropology} \\ Box 353100 \\ University of Washington  \\ \vspace{-0.05in} Seattle \textsc{wa} 98195-3100}  
  
\end{minipage}
\hfill     
%\begin{minipage}[t]{0.0in}
% dummy (needed here)
%\end{minipage}
\hfill
\begin{minipage}[t]{1.7in}
  \flushright \footnotesize  \addressblock \myphone \, \faPhone \\ 
  {\scriptsize  \texttt{\href{http://twitter.com/benmarwick}{\mytwitter}} \, \faTwitter }  \\ 
  {\scriptsize  \texttt{\href{mailto:\myemail}{\myemail}} \, \faEnvelope} \\
  {\scriptsize  \texttt{\href{\myweb}{\myweb}} \, \faGlobe}
\end{minipage}

\medskip

%% Name 
\noindent{\LARGE\scheader \textsc{ben marwick}}
\reversemarginpar

\bigskip       

%% Rearrange these sections according to requirements... for tenure it's got to be like this:
%
% Education -- institutions, degrees granted, dates
% Ph.D. dissertation title
% Employment -- institutions (including UW), positions, dates
% UW committees and other duties
% Research projects, grants, contracts -- funding agencies, dates, amounts of funding, individual's role (PI, co-PI, other)
% Professional offices and awards, with dates
% Talks, papers, or presentations -- dates, type of presentation (invited, contributed, and/or refereed)

\marginhead{\sffamily {{\vskip -0.35em} education}}

\ind Ph.D., Archaeology and Natural History, Australian National University, 2008.

 \vspace{0.05in}
\begin{Spacing}{0.75}
\ind \hspace{0.35in} 
\footnotesize {Dissertation: \emph{\href{http://dx.doi.org/10.6084/m9.figshare.765252}{Stone artefacts and human ecology at two rockshelters in Northwest Thailand}}} \normalsize \vspace{0.05in}
 \end{Spacing}

\ind M.A., Archaeology, University of Western Australia, 2002. 

\vspace{0.05in}
\begin{Spacing}{0.75} 
\ind \hspace{0.35in} \footnotesize Dissertation: \emph{\href{http://dx.doi.org/10.6084/m9.figshare.765251}{ Inland Pilbara Archaeology: A Study of Variation in Aboriginal Occupation over Time and Space on the Hamersley Plateau}} \normalsize \vspace{0.05in}
 \end{Spacing}

\ind B.A. (Hons.), First Class, University of Western Australia, 1999.

\bigskip

\marginhead{\sffamily {{\vskip -0.35em} PhD thesis \newline title}}

\ind Marwick, B. 2008. \emph{\href{http://dx.doi.org/10.6084/m9.figshare.765252}{Stone artefacts and human ecology at two rockshelters in Northwest Thailand.}} Archaeology and Natural History Department, Australian National University, Canberra


\bigskip

\marginhead{\sffamily  {{\vskip -0.35em} employment}}

\ind Professor in Archaeology,  University of Washington, 2021--present.

\ind Associate Professor in Archaeology,  University of Washington, 2015--2021.

\ind Senior Research Fellow (ARC Future Fellow),  University of Wollongong, 2015--2018.    

\ind Assistant Professor in Archaeology,  University of Washington, 2008--2015.     

\ind Consultant Archaeologist, Huonbrook Environment and Heritage, South Australia, 2006--2008.

\ind Research Associate, Society, State and Governance in Melanesia Project and The Pacific Centre, Research School of Pacific and Asian Studies, The Australian National University, 2006--2008.


\bigskip

\marginhead{\sffamily {\vskip 0.35em}UW \newline committees \newline \& other duties}

\medskip

\noindent\emph{Service within the Anthropology Department \vspace{0.01in}}

\medskip

\ind Web Site Committee, 2010-2013

\ind Undergraduate Curriculum Coordination Committee, 2013-current

\ind Sub-faculty Appointments Committee, 2012-2013

\ind Honours Committee, 2012-2013

\ind Honours Coordinator, 2009-2010

\ind Self-Sustaining MA Degree Program in Public Anthropology Planning Committee, 2009-2010

\medskip

\noindent\emph{Service within the Archaeology Program \vspace{0.01in}}

\medskip

\ind Archaeology Program Undergraduate Archaeological Sciences Option Coordinator, 2013-present

\ind Archaeology Program Teaching Schedule Coordinator, 2013-present

\ind Archaeology Program Graduate Comprehensive Exam Coordinator, 2012-present

\ind Archaeology Program Seminar Coordinator, 2009-2013

% \vspace{0.05in}
 \medskip
 
 \newpage

\noindent\emph{Service elsewhere on campus\vspace{0.01in}}

\medskip

\ind Chair, Faculty Council on Research, 2019-current

\ind Facilitator, Center for Teaching and Learning Faculty and Professional Learning Community 'Engaging Students in Larger Classes', Spring Quarter 2012 and Winter Quarter 2013

\ind Adjunct Curator, Archaeology Department, Burke Museum, 2010-current

\ind Affiliate faculty, UW Quaternary Research Center, 2010-current

\ind Affiliate faculty, UW Center for Statistics and Social Sciences, 2009-current

\ind Affiliate faculty, UW Center for Southeast Asian Studies, 2009-current

\ind Affiliate faculty, UW IGERT Program in Evolutionary Modeling, 2010-2013

\ind UW Center for Southeast Asian Studies MA Program admissions committee, 2009-2010

%% Service

%% \marginhead{\sffamily {\vskip 0.35em} service to the \newline profession}

\medskip

\noindent\emph{Editorial service\vspace{0.01in}}

\medskip

\ind Co-editor (with Peter Lape), Bulletin of the Indo-Pacific Prehistory Association, 2009-2013

\ind Associate Editor, Journal of World Prehistory (2008-present)

\ind Associate Editor, Scientific Data (2014-present)

\ind Associate Editor, Palgrave Communications (2019-present)

\vspace{0.05in}

\noindent\emph{Manuscript reviews\vspace{0.01in}}

\medskip

\noindent Journal of Archaeological Science, Journal of Human Evolution, American Antiquity, Current Anthropology, Journal of Archaeological Research, Australian Archaeology, Archaeology in Oceania, Journal of Geochemical Exploration, National University of Singapore Press, ANU Press, Australian Museum Press, C.N.R.S (National Centre for Scientific Research)

 \vspace{0.05in}
 \medskip

\noindent\emph{Grant proposal reviews\vspace{0.01in}}

\medskip

\noindent National Science Foundation (2008, 2009, 2013, 2018), UW Royalty Research Fund (2010), Australian Research Council (2015, 2017)


\bigskip

\marginhead{\sffamily {{\vskip 0.1em} research \newline projects  \newline \& grants}}
\medskip

\ind  Marwick, B. ``Geoarchaeological Research of sediment samples from Angkor, Cambodia" UW Quaternary Research Center (awarded Mar 2019, TDC USD 6144)

\ind  Marwick, B. ``Palaeolithic syntheses in East and Southeast Asia" Wenner Gren Foundation (awarded Mar 2019, TDC USD 30,000)

\ind  Marwick, B. ``New Palaeolithic archaeology excavations at Padahlin Cave, Myanmar" National Geographic Society (awarded Nov 2018, TDC USD 28,560)

\ind  Marwick, B. `Palaeolithic archaeology in Myanmar and western Thailand" Australian Research Council Future Fellowship (awarded Sept 2014, TDC AUD 640,000)

\ind  Louys, J., B. Marwick, J. Stevenson, G. Price. ``Super-volcanos and the human evolutionary trajectory: examining the ecological impacts of the Toba eruption on late Pleistocene hominids and fauna." Leakey Foundation (awarded April 2014, TDC USD 12,552)

\ind  Marwick, B. Deutscher Akademischer Austausch Dienst, Research Visit Travel Grant (awarded 2014, sole PI, TDC USD 6,000)

\ind  Marwick, B. ``Micromorphology at key archaeological sites at the end of the arc of modern human dispersal", UW Royalty Research Fund, (awarded 2014, sole PI, TDC USD 31,005)

\ind  Marwick, B and S. Kwak. ``Long-term chronology of subsistence and the role of intensive rice agriculture in the central part of the prehistoric Korean peninsula", National Science Foundation Doctoral Dissertation Research Improvement Grant, (awarded 2013,  PI with graduate student, TDC USD 23,373)

\ind  Roberts, R., Mokhtar Saidin, Ben Marwick, Dong Truong Nguyen, Nigel Spooner, Regina DeWitt, Anatoly Rosenfeld, Zenobia Jacobs, Bo Li, Susanna Guatelli, C. Clarkson, Adam Brumm and Kat Szabó ``Out of Asia: unique insights into human evolution and interactions using frontier technologies in archaeological science".  Australian Research Council (awarded 2013, TDC AUD 3,182,338)

\ind Lewis, H. and B. Marwick,  University College Dublin Seed Funding. University College Dublin (awarded 2012, Co-PI, TDC EUR 5,200.00) 

\ind Marwick, B. Vice Provost’s International Faculty Exchange Grant (awarded 2012, PI, TDC USD 3500)

\ind  Marwick, B. ``Investigating modern human origins and early behavioural complexity in Australia", UW Office of Research Trans-Pacific Early Career Academic Fellowship (awarded 2011, sole PI, TDC USD 13,125)

\ind Marwick, B. UW Office of Global Affairs Herbert H. Gowen International Studies Award (awarded 2011, sole PI, TDC USD 2,500)

\ind O’Connor. S., J. Fenner, J. Stevenson, K. Dobney, B. Marwick, E. St Pierre and G. Larson ``The Archaeology of Sulawesi: A Strategic Island for Understanding Modern Human Colonization and Interactions Across our Region", Australian Research Council (awarded 2010, Co-PI, TDC AUD 1,757,919)

\ind Clarkson, C., B. Marwick, L. Wallis, M. Smith and R. Fullgar  ``Modern Human Origins and Early Behavioural Complexity in Australia and South East Asia", Australian Research Council (awarded 2010). (Co-PI, TDC AUD 1,189,790)

\ind Marwick, B. UW Centre for Statistics in the Social Sciences Undergraduate Research Grant (awarded 2010, sole PI, TDC USD 4,480)

\ind Marwick, B. UW Office of Global Affairs Herbert H. Gowen International Studies Award (awarded 2010, sole  PI, TDC USD 5,000)

\ind Marwick, B. UW Centre for Statistics in the Social Sciences Seed Grant (awarded 2009, sole PI, TDC USD 18,347)

\ind Marwick, B. Dorothy Cameron Prize for best archaeological publication at The Australian National University (awarded 2008, AUD 500)

\ind Marwick, B. Postgraduate Research Grant, Australian Institute of Nuclear Science and Technology (awarded 2007, sole PI, TDC AUD 5,000)


\bigskip

\marginhead{\sffamily {{\vskip -0.35em} professional \newline awards}}

\ind Fellow, Technology Teaching Institute, UW Center for Teaching and Learning (awarded 2013, sole PI, TDC USD 7,000)

\ind Post-Doctoral Fellow, ACLS/Luce Foundation (Awarded 2010, sole PI, TDC USD 35,000)

\bigskip

\newpage

%% Talks and presentations - special invited talks

\marginhead{\sffamily {\vskip 0.3em} talks \& \newline presentations}

\medskip

\noindent\emph{Invited scholarly talks (since 2009)\vspace{0.01in}}

\medskip

\ind Marwick, B. 2018. ``Adaptations to sea level change and transitions to agriculture at Khao Toh Chong rockshelter, Peninsular Thailand", Invited presentation, Washington State University, 18 Jan 2018

\ind Marwick, B. 2017 ``Madjedbebe, Australia's earliest archaeological site Site formation, artefact movement and human behaviour", Invited workshop presentation at the ‘Using Archaeological Sciences to Address Human Behaviour’ workshop, University of Tübingen, Germany, 12-17 Nov, 2017

\ind Marwick, B. 2017. ``rrtools: writing reproducible research", Invited workshop presentation presented at the DataONE WholeTale ‘Prov-a-thon’, University of Illinois, August 31–September 1, 2017

\ind Marwick, B. 2017. href{https://github.com/benmarwick/berlinsummerschoolkeynote}{``Extracting Sunbeams Out of Cucumbers: Why Archaeology Isn't a Science, and How It Can Become One"} Keynote presentation, Summer School on Computational Archaeology, Freie Universität Berlin, 17-21 July 2017

\ind Marwick, B. 2017. ``A Practising Academic’s View of Field Data Collection Technologies." WOrkshop on Archaeological Database Evaluation, Cotsen Institute of Archaeology, UCLA,  Jun, 9, 2017. Invited Presentation.

\ind Marwick, B. 2017. ``Reproducible Research Compendia via R packages." ZüKoSt: Seminar on Applied Statistics, ETH-Zurich,  Mar, 2, 2017. Invited Presentation.

\ind Marwick, B. 2017. `Using R in the development sector in Myanmar and Vietnam: Benefits, challenges, and impacts of open source tools for working with data." Mathematical and Data Sciences for Development Forum, ETH-Zurich,  Mar, 1, 2017. Invited Presentation.

\ind Marwick, B. 2017. ``The Galisonian program, hard cores, and mirror recursion in Archaeological science." Conference on Critical Perspectives on the Practice of Digital Archaeology, Harvard University,  Feb, 3, 2017. Keynote Presentation.

\ind Marwick, B. 2016. ``Geoarchaeology of tektites in mainland Southeast Asia." Symposium on Anh Khe Archaeology, Institute of Archaeology, Vietnam, Oct, 30, 2016. Invited Presentation.

\ind Marwick, B. 2016. \href{https://github.com/benmarwick/UOW-NIASRA-2016-talk}{``Computational reproducibility in research and teaching: A case study from archaeology."} University of Wollongong National Institute for Applied Statistics Research Australia, May, 25, 2016. Invited Presentation.

\ind Marwick, B. 2016. ``Moving on from Movius: Recent research in Pleistocene archaeology in Myanmar." Monash University Archaeology Department, Melbourne, May, 17, 2016. Invited Presentation.

\ind Marwick, B. 2016. \href{https://github.com/benmarwick/Monash-Wombat-2016-talk}{``Computational Reproducibility in Archaeological Research: Basic Principles and a Case Study of Their Implementation."} Monash University Econometrics and Business Statistics, Melbourne, May, 17, 2016. Invited Presentation.

\ind Marwick, B. 2016. ``Mau A and the Hoabinhian-Sonvian transition." Vietnam Institute of Archaeology, Hanoi, Vietnam, 6 May 2016. Invited Presentation.

\ind Marwick, B. 2015. ``Three styles of Darwinian evolution in the analysis of stone artefacts Which one to use in East/Southeast Asia?" Department of History, Kyung Hee University, Seoul, Korea, Mar 3, 2015. Invited Presentation.

\ind Marwick, B. 2014. ``Palaeolithic archaeology in mainland Southeast Asia". Department of Archaeology, University of Yangon, Myanmar. Jan 19, 2014. Invited Presentation.

\ind Marwick, B. 2013. ``Research Report on Middle Holocene Research in mainland Southeast Asia". Southeast Asia Archaeology Workshop, University of Hawaii-Manoa, April 2013. Invited Presentation.

\ind Marwick, B. 2013 ``The Hoabinhian of Southeast Asia and its relationship to global Pleistocene lithic technologies". International Symposium on the Palaeolithic Cultures in Taiwan and Its Surrounding Areas, Academia Sinica, Taitung, Taiwan, 29-31 March 2013.  Invited Presentation.

\ind Marwick, B. 2013 ``Pleistocene lithic technologies in Mainland Southeast Asia: How does the Hoabinhian fit in?"  International Conference on Recent Advances in the Archaeology of East and Southeast Asia. University of Wisconsin-Madison, Wisconsin, USA, March 15-16, 2013. Invited Presentation.

\ind Marwick, B. 2012 ``Cultural Phylogenetics of Religion using Thai and Lao Bronze Buddha Statues". Department of Archaeology, University College Dublin, Ireland. Nov 22, 2012. Invited Presentation.

\ind Marwick, B. 2009 ``Pleistocene Exchange Networks as Evidence of Language Evolution".  UBC/SFU Human Evolution, Culture and Cognition seminar series, UBC, Vancouver, Canada. Invited presentation

\bigskip

%% Talks and presentations - regular contributed papers

% \marginhead{\sffamily {\vskip 0.3em} contributed \newline scholarly \newline talks \& \newline presentations \newline (since 2008)}

\noindent\emph{Contributed scholarly talks \& presentations (since 2009)\vspace{0.01in}}

\medskip

\ind  Marwick, B. et al. 2017 ```Stratigraphy, site formation, and artefact taphonomy at Madjedbebe, NT"  presentation at the Australian Archaeological Association Conference, 2017, Melbourne, Australia, 6-8 Dec 2017.

\ind  Yue H., B. Marwick, W. Huang, J. Zhang and B. Li  2017. ``Middle Pleistocene Lithic Technology at Guanyingdong, Guizhou Province, China".  Peer-reviewed presentation at the Society of American Archaeology conference, Apr 2017, Vancouver.

\ind  Marwick, B. 2017. ``Moving on from Movius: Recent research in Pleistocene archaeology in Myanmar".  Peer-reviewed presentation at the Society of American Archaeology conference, Apr 2017, Vancouver.

\ind Conrad, C., C. Higham, M. Eda and B. Marwick 2016 ``Hunter-Gatherer Foraging Adaptations at Spirit Cave and Banyan Valley Cave (Mae Hong Son Province, Northwest Thailand) During the Late Pleistocene and Holocene." Peer-reviewed presentation at the SEAMO-SPAFA conference, Apr 2016, Bangkok, Thailand.

\ind Kretzler, I., J. Whittaker and B. Marwick 2015. ``Grand Challenges vs Actual Challenges: Text Mining Small and Big Data for Quantitative Insights".  Peer-reviewed presentation at the Society of American Archaeology conference, Apr 2015, San Francisco.

\ind Marwick, B. 2015 {\href{https://github.com/benmarwick/SAA2015-Open-Methods}{``Reproducible Research in Archaeology: Basic Principles and Common Tools"}. Peer-reviewed presentation at the Society of American Archaeology conference, Apr 2015, San Francisco.

\ind De Boer, D., Zara Steinhart , Ben Marwick, David Bulbeck and Sue O'Connor. 2015 ``Stone Artefacts from Southeast Sulawesi: Technology Beyond the Toalean". Peer-reviewed presentation at the Society of American Archaeology conference, Apr 2015, San Francisco.

\ind Marwick, B. and Maloney, T. 2014 {\href{https://github.com/benmarwick/marwick-and-maloney-saa2014}{``Elliptical Fourier Analysis Unravels a Palimpsest of Point Technology  in the Kimberley, Western Australia"}}. Peer-reviewed presentation at the Society of American Archaeology conference, Apr 2014, Austin.

\ind Kretzler, I. and B. Marwick 2014 {\href{https://github.com/benmarwick/kretzler-and-marwick-saa2014}{``Understanding Archaeological History Through Textual Macroanalysis: The Role of Feminism in Gender Research"}. Peer-reviewed presentation at the Society of American Archaeology conference, Apr 2014, Austin.

\ind Sonethongkham, S. and B. Marwick 2014 ``Variation in cores excavated from four rockshelter sites in Luang Prabang Province, Lao PDR". Peer-reviewed presentation at the Indo-Pacific Prehistory Association conference, Jan 2014, Siam Reap.

\ind Marwick, B.  and S. Sonethongkham 2014 ``Patterns in Holocene flaked artefact technology from excavations in Luang Prabang Province, Lao PDR" . Peer-reviewed presentation at the Indo-Pacific Prehistory Association conference, Jan 2014, Siam Reap.

\ind Marwick, B. C. Clarkson, S. O’Connor and S. Collins 2014 ``Pleistocene lithic technology from East Timor: remarkable long term stability in technological behaviours". Peer-reviewed presentation at the Indo-Pacific Prehistory Association conference, Jan 2014, Siam Reap.

\ind Hayes, E., R. Fullagar, C. Clarkson, M. Smith and B. Marwick 2014 ``Pleistocene evidence for seed grinding in Australia". Peer-reviewed presentation at the Indo-Pacific Prehistory Association conference, Jan 2014, Siam Reap.

\ind Steele, T, A. Mackay, K. Fitzsimmons, B. Marwick, J. Orton, S. Schwortz, M. Stahlschmidt 2014 ``Varsche Rivier 003: A Middle Stone Age site with Still Bay and Howiesons Poort assemblages in southern Namaqualand, South Africa." Peer-reviewed presentation at the European Society for the study of Human Evolution (ESHE). 18-20 September, 2014, Florence, Italy.

\ind Clarkson, C., M. Smith, B. Marwick, R. Fullagar, T. Manne, K. Lowe, X. Carah,  A. Florin, A.  Fairbairn, C. Pardoe and L. Wallis 2014 ``Excavations at Malakunanja II, northern Australia",  Peer-reviewed presentation at the Indo-Pacific Prehistory Association conference, Jan 2014, Siam Reap.

\ind Clarkson, C., M. Smith, R. Fullagar, B. Marwick, T. Manne, L. Wallis, Z. Jacobs, R. Roberts, K. Lowe, A. Florin, X. Carah, C. Pardoe and E. Hayes 2013 ``Report on new research at Madgedbebe (Malakunanja II)" . Peer-reviewed presentation at the Australian Archaeological Association Annual Conference, Dec 2013, Coff’s Harbour.

\ind Wallis, L., C. Pardoe, T. Manne, K. M. Lowe, J. Matthews, C.   Clarkson, B. Marwick, R. Fullagar and M. Smith 2013 ``Much more than just an old site: Human remains from the 2012 excavations at Madjedbebe, western Arnhem Land, Australia". Peer-reviewed presentation at the Australian Archaeological Association Annual Conference, Dec 2013, Coff’s Harbour.

\ind Lowe, K. L. A. Wallis, C. Pardoe, B. Marwick C. Clarkson, T. Manne, M. Smith and R. Fullagar. 2013 ``Ground-penetrating radar, GIS and burial practices in western Arnhem Land, Australia". Peer-reviewed presentation at the Australian Archaeological Association Annual Conference, Dec 2013, Coff’s Harbour.

\ind Hayes, E.  R. Fullagar, C. Clarkson, B. Marwick, M. Smith, L. Wallis and T. Manne, 2013 ``Pleistocene evidence for seed grinding in Australia". Peer-reviewed presentation at the Australian Archaeological Association Annual Conference, Dec 2013, Coff’s Harbour.

\ind Florin, A., C. Clarkson, A. Fairbairn, R. Fullagar, B. Marwick, L. Wallis, X. Carah, T. Manne and M. Smith. 2013 ``Archaeobotanical investigations into plant food use at Madjedbebe (Malakunanja II)". Peer-reviewed presentation at the Australian Archaeological Association Annual Conference, Dec 2013, Coff’s Harbour.

\ind Marwick, B. E. Hayes and C. Clarkson 2013 ``Movement of lithics by trampling: An experiment in the Malakunanja II sediments". Peer-reviewed poster at the Society of American Archaeology Annual Meeting, April 3-7, Honolulu, Hawaii. 2013

\ind Macy, K., B. Marwick, B. S. Hanyu, C. Conrad and A. Mackay 2013 ``Where did the dirt come from? Geochemical sediment tracing at Klipfonteinrand rockshelter, western South Africa". Peer-reviewed poster at the Society of American Archaeology Annual Meeting, April 3-7, Honolulu, Hawaii. 2013

\ind Bulbeck, D., B. Marwick, S. O'Connor, F. Aziz, A. Calo, J. Fenner, B. Hakim, A. Oktaviana and E. St Pierre 2013 ``Archaeology of Lake Towuti: Survey and Excavation in South Sulawesi". Peer-reviewed poster at the Society of American Archaeology Annual Meeting, April 3-7, Honolulu, Hawaii. 2013

\ind Marwick, B., P. Hiscock, P. Hughes and M. Sullivan. 2013 ``The use of mobile Geographic Information Systems for intensive archaeological survey in arid northeastern South Australia". Peer-reviewed presentation at the Computer Applications and Quantitative Methods in Archaeology Conference, University of Western Australia, Perth, Australia, March 18-21, 2013

\ind Marwick, B. ``How archaeologists explain (and how they ought to): Theory and practice of explanation in Australian archaeology". Peer-reviewed presentation at the Australian Archaeological Association, Wollongong, Australia, Dec 2012.

\ind Marwick, B, Anna Cohen, S. Kwak, K. Macy and A. Cowan 2011 ``Geoarchaeology of Iron Age ceramics at Tham Sua Cave, Northern Lao PDR". Peer-reviewed poster at the Society of American Archaeology Annual Meeting, Sacramento, CA.

\ind Marwick, B, T. Field, J. Martinez, P. Przystupa, K. Abbott, K. Macy, L. Minchk, P. Hughes and M. Sullivan 2010. ``Landscape archaeology in arid northeast South Australia: Preliminary investigations into relationships between site distributions, surface geology, water sources and wind patterns". Peer-reviewed poster at the Australian Association of Archaeology Annual Conference, Bateman’s Bay, New South Wales.

\ind Marwick, B, S. O’Connor and S. Collins 2010. ``Pleistocene-aged stone artefacts from Jerimalai, East Timor: Early modern human technology in island Southeast Asia". Peer-reviewed presentation at the 13th International Conference of the European Association of Southeast Asian Archaeologists (EurASEAA), Berlin, Germany.

\ind Marwick, B and Vanpheng Keopanna 2010. ``A pilot study of Geometric morphometric analysis of bronze Buddha craft traditions in the Lao PDR". Peer-reviewed presentation  at the 13th International Conference of the European Association of Southeast Asian Archaeologists (EurASEAA), Berlin, Germany.

\ind Marwick, B, 2010. ``Do Model-based explanations suit archaeology better than Inference to the Best Explanation?" Peer-reviewed presentation at the Society of American Archaeology Annual Meeting, St Louis, MO.

\ind Marwick, B, Ariel Auerbach and A. Cowan 2009. ``Geoarchaeology at Tham Sua Cave, Luang Prabang Province, Lao PDR". Peer-reviewed presentation at the Indo-Pacific Prehistory Association Congress, Hanoi, Vietnam.

\ind Brockwell, S., B. Marwick, P. Bourke, P. Faulkner, R. Willan 2009. ``Climate change records from North Australian cultural midden deposits: evidence from a pilot study of oxygen isotopes in mollusc shells". Peer-reviewed presentation  at the Australian Archaeological Association Annual Conference, Flinders University, Adelaide, South Australia.

% \ind Brockwell, S., P. Bourke, P. Faulkner, B. Marwick, B. Meehan. 2008. ``Climate Change and Behavioural Variability in Coastal Northern Australia". Peer-reviewed presentation at the World Archaeological Congress, Dublin,  Jul 2008.

% \ind Marwick, B. P. Hughes, P Hiscock, M. Sullivan, O. Macgregor 2008. ``Mobile Geographic Information Systems for Large-scale Intensive Archaeological Survey of Stone Artefact Scatters in Arid Australia". Peer-reviewed presentation at the Australian Archaeological Association Annual Conference, Noosa Dec 3-6 2008

% \ind Marwick, B. 2008. ``Measuring pre-processing in unretouched flaked stone artefact assemblages". Peer-reviewed presentation at the Australian Archaeological Association Annual Conference, Noosa Dec 3-6, Australia

% \ind Marwick, B. 2008. ``Hoabinhian flaked stone artefact palaeoeconomics and palaeoecology". Peer-reviewed presentation at the Society of American Archaeology, Vancouver, March 2008, Canada

% \ind Marwick, B. 2008 .``Ceramics of the three tributaries Luang Prabang, Lao PDR". Peer-reviewed presentation at the  Society of American Archaeology, Vancouver, March 2008, Canada

 \bigskip

 %% Talks at UW - scholarly

% \marginhead{\sffamily {\vskip 0.3em} invited  \newline scholarly \newline  talks \& \newline presentations \newline within UW}

% \noindent\emph{Invited scholarly talks \& presentations within UW \vspace{0.01in}}
%
% \medskip
%
% \ind Marwick, B. 2014. {\href{https://github.com/benmarwick/UW-eScience-reproducibility-social-sciences}{``Reproducible Research: A View from the Social Sciences".}}  UW eScience Institute, 9 Apr 2014. Invited presentation.
%
% \ind Marwick, B. 2014. {\href{https://github.com/benmarwick/CSSS-Primer-Reproducible-Research}{``Reproducible Research: A Primer for the Social Sciences".}}  UW Center for Statistics and the Social Sciences, 12 Mar 2014. Invited presentation.
%
% \ind Marwick, B. and I Kretzler 2014. {\href{https://github.com/benmarwick/Data-Science-at-UW-Poster}{``Text mining JSTOR: Quantitative approaches toward histories of science".} Poster presented to the UW eScience Institute Data Science Event, 7 Feb Oct 2014. Contributed presentation.
%
% \ind Marwick, B. 2014. ``Computational methods for the analysis of unstructured text". UW Department of Linguistics  Computational Linguistics Group, 4 Feb 2014. Invited presentation.
%
% \ind Marwick, B. ``Pleistocene language evolution and raw material transfers". UW Linguistics Department. 22 Nov 2013. Invited colloquium
%
% \ind Marwick, B. 2013. ``Mobile GIS for intensive archaeological survey of Holocene forager sites in arid South Australia". GIS Day Lightning Talk,  for the UW Geography Department 20 Nov 2013. Invited presentation.
%
% \ind Marwick, B. 2012 ``Cultural Phylogenetics of Religion using Thai Bronze Buddha Statues". UW Centre for Statistics and Social Sciences, 4 April 2012. Invited presentation.
%
%  \bigskip
%
%  %% Talks at UW
%
% % \marginhead{\sffamily {\vskip 0.3em} invited talks \newline on teaching \& learning \newline within UW}
%
% \noindent\emph{Invited talks on teaching and learning within UW \vspace{0.01in}}
%
% \medskip
%
%
% \ind Marwick, B. and I. Kretzler 2013. ``How to get students to read a lot. Distant reading in an archaeological theory seminar". Ignite presentation for the UW Center for Teaching and Learning eText workshop 24 Oct 2013 Invited presentation.
%
% \ind Marwick, B. 2013. ``Engaging large classes with technology, especially Canvas." Workshop presentation for the UW-IT Academic Services and the Center for Teaching and Learning, 7 Mar 2013.  Invited presentation.
%
% \ind Marwick, B. 2012. ``Engaging large classes with technology, especially Canvas." Presentation for the UW Center for Teaching and Learning's Large Class Collegium, 29 Aug 2012.  Invited presentation.

\medskip


\bigskip


\end{document}
