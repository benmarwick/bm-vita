%%% A template to produce a nice-looking Curriculum Vitae.
%%% Originally from Kieran Healy <kjhealy@gmail.com>
%%% Kieran's source is at http://kjhealy.github.com/kjh-vita
%%%
%%% This source code has been adapted by 
%%% Ben Marwick <benmarwick@gmail.com> and is largely
%%% determined by the instructions here:
%%% \{https://admin.artsci.washington.edu/promotion-and-tenure-documentation#curriculum}
%%%
%%% questions: where to put on campus presentation on scholarly and teaching topics?
%%% ------------------------------------------------------------------------
%%% Requirements for this docuemnt that are included
%%%  in (or otherwise easily obtained) a modern tex distribution:
%%% ------------------------------------------------------------------------
%%% xelatex (I'm using MiKTeX & Texmaker)
%%% fontspec.sty
%%% hyperrref.sty
%%% xunicode.sty
%%% color.sty
%%% url.sty
%%% fancyhdr.sty
%%% memoir.cls
%%% fontawesome.sty
%%% gitinfo.sty
%%% 
%%% 
%%% ------------------------------------------------------------------------
%%% Requirements from https://github.com/kjhealy/latex-custom-kjh
%%% ------------------------------------------------------------------------
%%% org-preamble-xelatex.sty
%%% memoir-article-styles.sty
%%%
%%% 
%%% ------------------------------------------------------------------------
%%% Note
%%%------------------------------------------------------------------------
%%% Because this is a hand-tweaked file, be on the look out for \medksip, 
%%% \bigskip and \newpage commands here and there, which are used to balance
%%% the layout or avoid widows & orphans, etc. You should of course add or 
%%% remove these as needed.
%%%------------------------------------------------------------------------

\documentclass[11pt,article,oneside]{memoir}   
\usepackage{microtype}
\usepackage{org-preamble-xelatex} 
\usepackage{fontawesome,url}
\usepackage{setspace}
\usepackage[footinfo]{gitinfo}
\input{../gitHeadInfo.gin}

%%%------------------------------------------------------------------------
%%% Metadata
%%%------------------------------------------------------------------------

%% Change as needed.
\def\myauthor{Ben Marwick}
\def\mytitle{Vita}
\def\mycopyright{\myauthor}
\def\mykeywords{}
\def\mybibliostyle{plain}
\def\mybibliocommand{}
\def\mysubtitle{}
\def\myaffiliation{University of Washington}
\def\myaddress{Anthropology Department}
\def\myemail{bmarwick@uw.edu}
\def\myweb{http://faculty.washington.edu/bmarwick/ }
\def\myphone{(206) 552-9450}
\def\myfax{(206) 543-3285}
\def\mytwitter{@benmarwick}
\def\myversion{}
\def\myrevision{}


\def\myaffiliation{University of Washington}
\def\myauthor{Ben Marwick}
\date{} % not used (revision control instead)
\def\mykeywords{}


%%%------------------------------------------------------------------------
%%% Document
%%%------------------------------------------------------------------------

\begin{document}

%% Choose fonts for use with xelatex
%% Using free fonts with big glyph sets for maximum flexibility  
%% http://www.linuxlibertine.org/index.php?id=2&L=1
%% http://font.ubuntu.com/

\setromanfont[Mapping={tex-text}, 
	Numbers={OldStyle},
	Ligatures={Common}]{Linux Libertine}
\setsansfont[Mapping=tex-text,
	Ligatures={Common}, 
	Colour=AA0000]{Linux Biolinum}
\setmonofont[Mapping=tex-text,Scale=0.72]{Ubuntu} 

\newfontface\scheader[SmallCapsFont={Linux Libertine},SmallCapsFeatures={Letters=SmallCaps}]{Linux Libertine}

\newfontface\addressblock[Mapping={tex-text}, 
	Numbers={OldStyle},
	Ligatures={Common}]{Linux Libertine}


%%%------------------------------------------------------------------------
%%% Local commands
%%%------------------------------------------------------------------------

%% Marginal header
%% Note: as the document goes on you may need to introduce a (gradually increasing)
%% \vspace element to keep the marginal header pleasingly aligned with the first 
%% item in the body text. Like this: \marginhead{{\vskip 0.4em}Grants}, or 
%% \marginhead{{\vskip 0.8em}Service}. Experiment as needed.
\newcommand{\marginhead}[1]{\marginpar{\textsf{{\footnotesize\vspace{-1em}\flushright #1}}}}


%% [optional] custom ampersand (font consistent with the one chosen above)
\newcommand{\amper}{{\fontspec[Scale=.95,Colour=AA0000]{Linux Libertine}\selectfont\&\,}}

%% No bullets on labels
\renewcommand{\labelitemi}{~} 

%% Custom hanging indent for vita items
\def\ind{\hangindent=1 true cm\hangafter=1 \noindent}
%\def\ind{\hangindent=18pt\hangafter=1 \noindent}
\def\labelitemi{~}
\renewcommand{\labelitemii}{~}

%%%------------------------------------------------------------------------
%%% Page layout
%%%------------------------------------------------------------------------

% These lines will insert git revision info in the footer, using the gitinfo
% package---see docs for gitinfo package for details. Comment out this line
% if you're not using git.
\pagestyle{kjh}
\thispagestyle{kjhgit}

%%%------------------------------------------------------------------------
%%% Address and contact block
%%%------------------------------------------------------------------------
%\begin{minipage}[t]{2.95in}
% \flushright {\footnotesize 
% \href{http://depts.washington.edu/anthweb/}{Department of Anthropology} \\ Box 353100 \\ University of Washington  \\ \vspace{-0.05in} Seattle \textsc{wa} 98195-3100}  
%  
%\end{minipage}
%\hfill     
%%\begin{minipage}[t]{0.0in}
%% dummy (needed here)
%%\end{minipage}
%\hfill
%\begin{minipage}[t]{1.7in}
%  \flushright \footnotesize  \addressblock \myphone \, \faPhone \\ 
%  {\scriptsize  \texttt{\href{http://twitter.com/benmarwick}{\mytwitter}} \, \faTwitter }  \\ 
%  {\scriptsize  \texttt{\href{mailto:\myemail}{\myemail}} \, \faEnvelope} \\
%  {\scriptsize  \texttt{\href{\myweb}{\myweb}} \, \faGlobe}
%\end{minipage}

\medskip

%% Name 
\noindent{\LARGE\scheader \textsc{personal statement}}
\reversemarginpar

\bigskip       

% Personal statement for tenure
% http://admin.artsci.washington.edu/promotion-and-tenure-documentation#candidates

% The statement should not be longer than three to five pages.
%
% In the discussion of research or scholarship, a short essay is more effective than an annotated list of works. The candidate's research contributions might be described in the broader context of the discipline as a whole, explaining how his or her research agenda fits into the discipline and then how particular scholarly or creative contributions fit into this agenda. The essay should also include the candidate's statement of future directions and how these connect to previous and current work, in order to give a sense of the trajectory of the work.
%
% The personal statement should contain as well a discussion of the candidate's teaching experience, with an overview of the candidate's goals, a review of successes and failures, reflections on these experiences, and thoughts of what lies ahead. Lastly, the candidate should describe any significant service contributions.

% ideas... 
% http://www.insidehighered.com/advice/2010/11/10/narrative
% http://www.slideshare.net/UO-AcademicAffairs/writing-a-tenure-statement-2011


%% set paragaph breaks to be blank lines and don't indent first line
\nonzeroparskip
\setlength{\parindent}{0pt}

\bigskip       

\marginhead{\sffamily {{\vskip 0.5em}  introduction}}

I am an archaeologist interested in human behavior, technology and ecology. My approach to these themes is motivated by models and methods from the evolutionary sciences which I seek to adapt to better understand the human past. My passion as a scholar is to apply evolutionary approaches to address questions of technological variation and ecological adaptation, cultural change and cultural transmission ranging from the deep past to recent times. Unlike many archaeologists motivated by evolutionary theory, I am fascinated by the challenges of primary data collection, and my fieldwork in mainland Southeast Asia and Australia supplies the empirical content of my research. Fundamental to these activities is my deep concern for the social relevance and disciplinary integrity for archaeology. What sets my work apart from others is this combination of evolutionary theory applied to primary data collected in Southeast Asia in collaborative arrangements that prioritize training and capacity building.  

In what follows, I will narrate my research trajectory since arriving at the University of Washington in 2008 and explain how the characteristics that distinguish my research – namely, its use of evolutionary approaches and its focus on archaeological problems of Southeast Asia – contribute to my teaching and service. 

\bigskip  
%% \newpage   

\marginhead{\sffamily  {{\vskip 0.3em} research \newline trajectory}}

I began my research career investigating questions about Australian Aboriginal hunter-gatherer adaptation during the Late Pleistocene and Holocene periods (Marwick {\href{http://hdl.handle.net/1885/42085}{2002a}}, {\href{http://dx.doi.org/10.6084/m9.figshare.765251}{2002b}}, {\href{http://faculty.washington.edu/bmarwick/PDFs/Marwick_2005_Marillana_A.pdf}{2005c}}, \href{http://faculty.washington.edu/bmarwick/PDFs/Marwick_2009_AO_Pilbara.pdf}{2009}, \href{http://faculty.washington.edu/bmarwick/PDFs/Hughes_et_al_2011_JASSA.pdf}{Hughes et al. 2011}, \href{http://faculty.washington.edu/bmarwick/PDFs/Sullivan_et_al_2012_OSL_dates_ODX.pdf}{Sullivan et al. 2012}, \href{http://faculty.washington.edu/bmarwick/PDFs/Brockwell_et_al_2013_AA.pdf}{Brockwell et al. 2013)} . Motivated by the success of combining stone artefact and geoarchaeological analyses in overcoming problems of a sparse archaeological record, I took up the challenge of  investigating prehistoric human-environment relations in tropical Southeast Asia, where the archaeological record is notoriously sparse for hunter-gatherers {\href{http://dx.doi.org/10.6084/m9.figshare.765252}{(Marwick 2008)}}. I have pursued this theme of prehistoric human-environment relations in three related projects.

First, I constructed a local palaeoclimate proxy from oxygen isotope values obtained from shell fish recovered from the same archaeological deposit as the stone artefacts {\href{http://faculty.washington.edu/bmarwick/PDFs/Marwick_and_Gagan_2011_QSR.pdf}{(Marwick and Gagan 2011)}}. This oxygen isotope curve provides the first evidence of a major change in conditions from the Pleistocene to the Holocene, consistent with evidence from similar records in China. We noted the absence of a signal of the Younger Dryas (YD) in our data, a key climatic event linked to the emergence of agriculture in many parts of the world. We used this evidence to engage in debates about the origins of agriculture in Southeast Asia. Our evidence indicated a likely origin in China where there is a strong YD signal, rather than agriculture developing locally in Southeast Asia.

Second, as a core member of the Middle Mekong Archaeology Project (MMAP) I excavated two rockshelters in the uplands of northern Laos to further investigate the question of how hunter-gatherers became farmers in mainland Southeast Asia. We predicted that if agriculture moved into Southeast Asia from China, we might find strong early signals if this new adaptation in Laos, due to the ease of movement facilitated by the Mekong.  The archaeological results, noted in our \textit{Antiquity} article {\href{http://antiquity.ac.uk/projgall/white/}{(White et al 2009})}, proved to be of limited relevance to my core interest in hunter-gatherers because the excavations recovered extensively bioturbated and only very recent (Iron Age) deposits. 

Third, in my \textit{Journal of Anthropological Archaeology} article I made a novel modification to standard behavioural ecological models  {\href{http://faculty.washington.edu/bmarwick/PDFs/Marwick_2013_JAA.pdf}{(Marwick 2013)}}. I began this analysis with the application of three standard behavioural ecological models that have been widely used to target the causes and consequences of behaviour and generate hypotheses that are testable with quantitative data. I found that when testing the predictions of the models with archaeological data, the three models did not give a consistent indication of the main variables influencing stone artefact assemblage production. Taking inspiration from Sewall Wright's idea of multiple optima in evolutionary biology, I revised the standard models to incorporate multiple behavioral optima and found a consistent fit with my data. This use of multiple optima is important because it shows how human-environment relationships can be modeled for edge cases where, at first glance, they do not appear to work.

\textbf{Human adaptation in tropical coastal and island environments} With my early work focused on inland and upland settings, I became curious about human adaptation in tropical coastal and island environments. I am now seeking to evaluate the usefulness of my model to understand the adaptations and evolutionary history of people with maritime adaptations. One of my specific historical and adaptive concerns is to understand transitions to agriculture in regions where past sea level changes – which others have argued was a driving factor in the domestication process –  had a profound effect on the landscape. Two ongoing projects directly relate to this question.

First, while a Luce/ACLS post-doctoral fellow during 2010-11, I initiated and lead a project with Thai collaborators to explore hunter-gatherer adaptations in coastal environments through survey, excavation in southern Thailand. Together with my Thai colleagues and US graduate students, we regularly present the results of our analysis at professional meetings and have one scholarly paper published {\href{http://faculty.washington.edu/bmarwick/PDFs/Conrad_et_al_2013_TNHMJ.pdf}{(Conrad et al. 2013)}}. So far our data show a clear signal of subsistence behaviours focused on fresh-water resources during the late Pleistocene and early Holocene shifting to mangrove swamp resources during the later Holocene as sea levels rose.  This is a much earlier shift than previously documented, challenging earlier work that recorded this dietary shift – often linked to the transition to agriculture – much later in the Holocene, complicating the narrative of agriculture as an introduction from China. 

Second, my recent work on Sulawesi, Indonesia was motivated by the question of whether my multiple optima model can be generalized to include tropical hunter-gatherer adaptations in island settings. As a co-PI with six collaborators on a project funded by the Australian Research Council (ARC), I directed two excavations on Sulawesi during 2012-13. These yielded archaeological sequences spanning the Late Pleistocene to recent times, providing data relevant to evaluating my model. We have presented some preliminary data at conferences and I am currently working on the geoarchaeologcal and lithics analyses.

\textbf{Long- and near-term evolutionary processes} To better understand these questions of human adaptation and the specific historical trajectories leading up to major events such as the transition to agriculture, I found it necessary to broaden my inquiry to investigate the longer term evolutionary processes of humans in Southeast Asia and Australia. My \textit{Quaternary International} article surveys the available evidence for hominin colonization of Southeast Asia and details three models that fit the data {\href{http://faculty.washington.edu/bmarwick/PDFs/Marwick_2009_QI.pdf}{(Marwick 2009b)}}. My current work is engaged in testing these models with three ongoing projects which are my focus for the near future. 

First, I am a co-PI with five colleagues on an ARC-funded project to investigate modern human origins and early behavioural complexity in Australia and Southeast Asia. The aim of this project is to collect and compare a large sample of early materials from three locations across Southeast Asia to Australia to more reliably date the appearance of modern humans and document the emergence of cultural diversity. In 2012 we excavated at Malakanunja II in northern Australia, one of Australia's oldest sites, in 2014 we will excavate at Jerimalai in East Timor, and in 2015 we will return to Thailand. I have been leading the geoarchaeological analysis and contributing to the lithic analysis of the Malakanunja II excavation. 

Second, I am a co-PI with three paleontologist colleagues on a Leakey Foundation grant to excavate archaeological deposits in Sumatra. We aim to investigate human colonization and adaptation relating to the Toba eruption 74 thousand years ago (fieldwork in Sumatra is scheduled for 2014). I am leading the archaeological research in this project.

Third, I am a co-PI with 13 colleagues on an ARC-funded project investigating hominin colonisation from India to Australia. My specific responsibility is to lead the survey, excavation and archaeological analysis in Myanmar and Thailand (fieldwork in Myanmar is scheduled for 2015).

Emerging from my application of evolutionary concepts to questions of human behaviour and adaptation in the remote past has been an interest in the potential of evolutionary approaches to provide insights into more recent historical and political processes.  Specifically, I have been pursuing attribute-based phylogenies and morphometric analyses of bronze Buddha statues from Laos and Thailand to better understand the flow of influence between different production centers. I have published some results of the phylogenetic analysis \href{http://faculty.washington.edu/bmarwick/PDFs/Marwick_2012_Buddha_cladistics.pdf}{(Marwick 2012)} and am currently working on the morphometric analysis and seeking a larger data set to broaden my scope. 

\textbf{Archaeological practice and public engagement} A parallel theme in my field and laboratory research activities has been the examination of questions of archaeological engagement and practice. My motivation for this work is to answer the questions of how to broaden the broader impacts of my archaeological research and how best to use archaeology as an agent of change outside of the discipline. I have been specifically concerned with deciphering the meanings of archaeology in popular culture, optimizing the relationship between archaeologists, the public, and local collaborators during fieldwork, and identifying the value of archaeologists communicating in public. Three of my scholarly products are related to these concerns of engagement and practice. 

First, I have defined how popular culture engages with archaeology in a study of blockbuster films  \href{http://faculty.washington.edu/bmarwick/PDFs/Marwick_2010_WA_Wall-E.pdf}{(Marwick 2009)}. I demonstrated a novel method for identifying archaeological themes in films that present a long-term narrative of the human experience but have no obvious archaeology content. The importance of this work is that it identifies points of engagement between archaeology and popular culture that have not been previously recognised. My high-enrollment first-year undergraduate course is organized around this research project. 

Second, as a Luce/ACLS fellow I developed a model of collaborative archaeology that has influenced my subsequent work {\href{http://faculty.washington.edu/bmarwick/PDFs/Marwick_et_al_2013_community.pdf}{(Marwick et al. 2013)}}. Specifically, I argued that community engagement can occur in three forms: syntactic (highly technical), semantic (generation of meanings) and pragmatic (resulting in practical intervention), and I list when and how each form might be best deployed. This model emerged during my work with MMAP where I organised extensive training of local archaeologists and museum workers {\href{http://faculty.washington.edu/bmarwick/PDFs/Marwick_et_al_2009_MMAP.pdf}{(Marwick et al. 2009)}}. I have continued with this practice in subsequent fieldwork. 

Third, I have developed computational methods for identifying controversy and topic trends in scholarly literature \href{https://github.com/benmarwick/JSTORr}{(Marwick 2013)} and used these tools to analyse how anthropologists use online communication \href{http://faculty.washington.edu/bmarwick/PDFs/Marwick_2013_DMAR.pdf}{(Marwick 2013)}. I determined that useful information about current controversies and research priorities can be reliably obtained from informal public correspondence such as Twitter messages. This work is important because it presents new methods for investigating the intellectual history and sociology of the discipline, and validates archaeologists' use of non-traditional media for communicating with the public. 

\bigskip     

\marginhead{\sffamily {{\vskip 0.3em} teaching}}

I teach classes at every level of the undergraduate and graduate program and the common foundation of these classes is engagement through active learning methods. Engagement has several dimensions in my teaching, not only of students with course content, but also of archaeology with the public, of theory with evidence, and of knowledge with practical application. 

\begin{description}

\item[Undergraduate research mentoring] I mentor 2-10 students per quarter in laboratory research. The students work side-by-side with me doing hands-on data collection using sediments and stone artefacts from my field work. They often present their results at the UW undergraduate research symposium and at international professional conferences and they have won prizes for best undergraduate archaeology writing at UW. I have also taught two five-week field schools (Australia, Thailand) for undergraduates that have a strong research focus. Several field school students have gone on to do follow-up lab research with me after the field school.

\item[Junior undergraduates] In my first-year course 'Archaeology in Film' we investigate representations of archaeology in popular films, culminating in the students (100-250 per quarter) making their own films (satisfying the university's Individual and Society requirement, and many students also satisfy their writing requirement with this class). I have embraced the unique challenges of engaging students from diverse backgrounds with the large lecture format, and enjoyed positive results using clickers, online activities, in-class group work and in-class low-stakes writing assignments.
  
\item[Mid-level undergraduates] I teach two regional survey classes (Australian Archaeology, Mainland Southeast Asian Archaeology) in a seminar style that actively engages students directly with primary research by having students analyse, remix and interpret data from scholarly journal articles. 
  
\item[Senior undergraduates] My main effort at this level relates to my role as the supervisor of the UW Geoarchaeology Lab. I teach geoarchaeology lab and seminar classes, where students learn skills that are vital for success in a career in archaeological science. These classes combine instruction in widely used earth sciences lab methods with collection of original data, data analysis and visualization and writing a report that demands intellectual engagement with current research problems in archaeology. We routinely use facilities held by other campus units in these research projects, notably a particle size analyser in Earth and Space Sciences that I co-purchased. After completing this class students often continue with undergratuate research projects under my supervision. 

\item[Graduate students] At the graduate level I lead a seminar class on archaeology and explanation that explores connections between scientific explanation as understood by philosophers of science and archaeological explanation. To gain insights the students collaborate using a combination of close reading and 'distant reading' or quantitative analysis of large amounts of text (using my \href{https://github.com/benmarwick/JSTORr}{JSTORr software}). I currently advise 12 students at UW and one elsewhere, chair committees for three students and have examined five students' PhD theses.
  
\end{description}
    
A unique focus of my research mentoring activity at all levels is encouraging students to adopt principles of reproducibility and sustainability that I have found to be beneficial in my own research. To this end I require students to record their work in open lab notebooks, to use software tools that are free and open source, to collect and store data in open formats, and to use literate computing workflows. I am enthusiastic about familiarizing students with the principles of reproducibility because requiring students to clearly document their research encourages them to think more clearly about what they are doing and reinforces what they are learning. Documenting the research process in a way that others can easily inspect also fosters a strong appreciation for research accountability and integrity early in the student's career. 


\bigskip     

\marginhead{\sffamily {{\vskip 0.3em} service}}

Within the archaeology program I have served annual terms as the curriculum coordinator, seminar coordinator, comprehensive exam coordinator and the coordinator of the Archaeological Sciences Option for undergraduates. As supervisor of the Geoarchaeology Lab I also maintain USDA permits enabling importation and storage of foreign specimens, and ensure the lab satisfies health and safety requirements. My key contributions to the Anthropology Department have been as the honors program coordinator (including reforms to promote faculty engagement with student research), the web site coordinator, and as a member of the the sub-faculty appointments committee. I led the transition to a new departmental webpage, and implemented a \href{https://docs.google.com/a/uw.edu/spreadsheet/ccc?key=0As7CmPqGXTzldF9DclQwY0NaU2JZNjBFQjg4RVdSdWc&single=true&gid=15&output=html}{cloud-based system to simplify collaborative coordination of teaching scheduling} within the department and the secure communication of the schedule to students. 

Within the UW I have contributed to the Center for Teaching and Learning (eg. seminars and workshops on large class engagement), the Burke Museum (eg. public talks), and the Centre for Southeast Asian Studies (eg. grant writing). Beyond the UW I have served as co-editor of the \textit{Journal of Indo-Pacific Archaeology} (with Peter Lape), on the editorial board of the \textit{Journal of World Prehistory}, as a peer reviewer for several international journals and book publishers (2-3 manuscripts per year), and funding agencies (1 grant reviewed per year). I contribute \href{https://github.com/benmarwick}{code} to open source research software maintained by the \href{http://ropensci.org}{rOpenSci} community and others.


\end{document}
